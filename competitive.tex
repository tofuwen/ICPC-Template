\documentclass[letterpaper]{article}
\usepackage{listings}
\usepackage{color}
\usepackage[left=1.2in,right=1.2in,top=1.2in,bottom=1.2in]{geometry}
\usepackage{titling}
\setlength{\droptitle}{2in}
\definecolor{dkgreen}{rgb}{0,0.6,0}
\definecolor{gray}{rgb}{0.5,0.5,0.5}
\definecolor{mauve}{rgb}{0.58,0,0.82}

\lstset{frame=tb,
		language=C++,
		aboveskip=3mm,
		belowskip=3mm,
		showstringspaces=false,
		columns=flexible,
		basicstyle={\small\ttfamily},
		numbers=none,
		%numberstyle=\tiny\color{gray},
		%keywordstyle=\color{dkgreen},
		commentstyle=\color{mauve},
		breaklines=true,
		breakatwhitespace=true,
		tabsize=3
}

\title{ICPC Algorithm Compilation}
\date{\today}
\author{Yuting Zhang}

\begin{document}
\maketitle
\thispagestyle{empty}
\newpage
\pagestyle{plain}
\tableofcontents
\setcounter{page}{1}
\pagenumbering{roman}
\newpage

\setcounter{page}{1}
\pagenumbering{arabic}
\section{Data Structures}
\subsection{Bitmasks}
\begin{lstlisting}
#define isOn(S, j) (S & (1 << j))
#define setBit(S, j) (S |= (1 << j))
#define clearBit(S, j) (S &= ~(1 << j))
#define toggleBit(S, j) (S ^= (1 << j))
#define lowBit(S) (S & (-S))
#define setAll(S, n) (S = (1 << n) - 1)

#define modulo(S, N) ((S) & (N - 1))   // returns S % N, where N is a power of 2
#define isPowerOfTwo(S) (!(S & (S - 1)))
#define nearestPowerOfTwo(S) ((int)pow(2.0, (int)((log((double)S) / log(2.0)) + 0.5)))
#define turnOffLastBit(S) ((S) & (S - 1))
#define turnOnLastZero(S) ((S) | (S + 1))
#define turnOffLastConsecutiveBits(S) ((S) & (S + 1))
#define turnOnLastConsecutiveZeroes(S) ((S) | (S - 1))
\end{lstlisting}

\subsection{Union-Find Disjoint Sets}
\begin{lstlisting}
class DisjointSets{
public:
	void addelements(int num){
		while (num--)
			s.push_back(-1);
	}
	int find(int elem) {
		return s[elem] < 0 ? elem : s[elem] = find(s[elem]); 
	}

	void setunion(int a, int b) {
		int root1 = find(a), root2 = find(b);
		int newSize = s[root1] + s[root2];
		if (s[root1] <= s[root2]){
			s[root2] = root1;
			s[root1] = newSize;
		}
		else{
			s[root1] = root2;
			s[root2] = newSize;
		}
	}

private:
    std::vector<int> s;
};
\end{lstlisting}
\newpage

\subsection{Segment Tree}
\begin{lstlisting}
// Segment tree for range sum queries.
struct segment_tree {
    vector<long long> st, lazy;
    const vector<long long> &A;
    size_t n;

    inline int left(int p) {
        return p << 1;
    }

    inline int right(int p) {
        return (p << 1) + 1;
    }

    void propagate(int p, int L, int R) {
        if (lazy[p] != 0) {
            if (L != R) {
                lazy[left(p)] += lazy[p];
                lazy[right(p)] += lazy[p];
            }
            st[p] += (R - L + 1) * lazy[p];
            lazy[p] = 0;
        }
    }

    void build(int p, int L, int R) {
        if (L == R)
            st[p] = A[L];
        else {
            build(left(p), L, (L + R) / 2);
            build(right(p), (L + R) / 2 + 1, R);
            st[p] = st[left(p)] + st[right(p)];
        }
    }

    long long update(int p, int L, int R, int i, int j, long long val) {
        propagate(p, L, R);

        if (L > j || R < i)
            return st[p];

        if (L >= i && R <= j) {
            lazy[p] = val;
            propagate(p, L, R);
            return st[p];
        }

        return st[p] = update(left(p), L, (L + R) / 2, i, j, val) + 
                       update(right(p), (L + R) / 2 + 1, R, i, j, val);
    }

    long long query(int p, int L, int R, int i, int j) {
        if (L > j || R < i)
            return 0;

        propagate(p, L, R);
        if (L >= i && R <= j)
            return st[p];
        return query(left(p), L, (L + R) / 2, i, j) +
               query(right(p), (L + R) / 2 + 1, R, i, j);
    }

    segment_tree(const vector<long long> &_A): A(_A) {
        n = A.size();
        st.assign(n * 4, 0);
        lazy.assign(n * 4, 0);
        build(1, 0, n - 1);
    }

    void update(int i, int j, long long val) {
        update(1, 0, n - 1, i, j, val);
    }

    long long query(int i, int j) {
        return query(1, 0, n - 1, i, j);
    }
};
\end{lstlisting}

\subsection{Fenwick Tree}
\begin{lstlisting}
#define LSOne(S) (S & (-S))

class FenwickTree {
private:
  vi ft;

public:
  FenwickTree() {}
  // initialization: n + 1 zeroes, ignore index 0
  FenwickTree(int n) { ft.assign(n + 1, 0); }

  int rsq(int b) {                                     // returns RSQ(1, b)
    int sum = 0; for (; b; b -= LSOne(b)) sum += ft[b];
    return sum; }

  int rsq(int a, int b) {                              // returns RSQ(a, b)
    return rsq(b) - (a == 1 ? 0 : rsq(a - 1)); }

  // adjusts value of the k-th element by v (v can be +ve/inc or -ve/dec)
  void adjust(int k, int v) {                    // note: n = ft.size() - 1
    for (; k < (int)ft.size(); k += LSOne(k)) ft[k] += v; }
};
\end{lstlisting}

\subsection{Treap}
\begin{lstlisting}
#include <iostream>
#include <cstdio>
#include <memory>
#include <vector>
#include <cstdlib>
#include <ctime>

using namespace std;

template<typename T>
class treap{
public:
    treap(){
        srand(time(0));
        root = nullptr;
    }

    void insert(const T& elem){
        insert(root, elem);    
    }

    void remove(const T& elem){
        remove(root, elem);
    }
    
private:
    struct node_t{
        T elem;
        shared_ptr<node_t> left, right;
        int priority;
    };

    shared_ptr<node_t> root;

    shared_ptr<node_t> rotateLeft(shared_ptr<node_t> node){
        shared_ptr<node_t> right = node->right, rightLeft = right->left;
        right->left = node;
        node->right = rightLeft;
        return right;
    }

    shared_ptr<node_t> rotateRight(shared_ptr<node_t> node){
        shared_ptr<node_t> left = node->left, leftRight = left->right;
        left->right = node;
        node->left = leftRight;
        return left;
    }

    void insert(shared_ptr<node_t>& node, const T& elem){
        if (node == nullptr){
            node = make_shared<node_t>();
            node->elem = elem;
            node->left = node->right = nullptr;
            node->priority = rand();
            return;
        }
        // We do not allow multiple keys with the same value
        if (node->elem == elem)
            return;

        if (node->elem > elem){
            insert(node->left, elem);
            if (node->priority < node->left->priority)
                node = rotateRight(node);
        }else{
            insert(node->right, elem);
            if (node->priority < node->right->priority)
                node = rotateLeft(node);
        }
    }

    void remove(shared_ptr<node_t>& node, const T& elem){
        if (node == nullptr)
            return;
        
        if (node->elem == elem){
            if (!node->left && !node->right)
                node = nullptr;
            // Keep rotating until the node to be deleted becomes a leaf node.
            else if (!node->left || (node->left && node->right && 
                node->left->priority < node->right->priority)){
                node = rotateLeft(node);
                remove(node->left, elem);
            }
            else{
                node = rotateRight(node);
                remove(node->right, elem);
            }
        }
        else if (node->elem > elem)
            remove(node->left, elem);
        else
            remove(node->right, elem);
    }
};
\end{lstlisting}

\section{Graph Theory}
\subsection{Topological Sort}
\begin{lstlisting}
void dfs2(int u) {    // change function name to differentiate with original dfs
  dfs_num[u] = DFS_BLACK;
  for (int j = 0; j < (int)AdjList[u].size(); j++) {
    ii v = AdjList[u][j];
    if (dfs_num[v.first] == DFS_WHITE)
      dfs2(v.first);
  }
  topoSort.push_back(u); }                   // that is, this is the only change

//inside int main()
  // make sure that the given graph is DAG
  printThis("Topological Sort (the input graph must be DAG)");
  topoSort.clear();
  dfs_num.assign(V, DFS_WHITE);
  for (int i = 0; i < V; i++)            // this part is the same as finding CCs
    if (dfs_num[i] == DFS_WHITE)
      dfs2(i);
  reverse(topoSort.begin(), topoSort.end());                 // reverse topoSort
  for (int i = 0; i < (int)topoSort.size(); i++)       // or you can simply read
    printf(" %d", topoSort[i]);           // the content of 'topoSort' backwards
  printf("\n");
\end{lstlisting}

\subsection{Articulation Points and Bridges}
\begin{lstlisting}
vi dfs_low;       // additional information for articulation points/bridges/SCCs
vi articulation_vertex;
int dfsNumberCounter, dfsRoot, rootChildren;

void articulationPointAndBridge(int u) {
  dfs_low[u] = dfs_num[u] = dfsNumberCounter++;      // dfs_low[u] <= dfs_num[u]
  for (int j = 0; j < (int)AdjList[u].size(); j++) {
    ii v = AdjList[u][j];
    if (dfs_num[v.first] == DFS_WHITE) {                          // a tree edge
      dfs_parent[v.first] = u;
      if (u == dfsRoot) rootChildren++;  // special case, count children of root

      articulationPointAndBridge(v.first);

      if (dfs_low[v.first] >= dfs_num[u])              // for articulation point
        articulation_vertex[u] = true;           // store this information first
      if (dfs_low[v.first] > dfs_num[u])                           // for bridge
        printf(" Edge (%d, %d) is a bridge\n", u, v.first);
      dfs_low[u] = min(dfs_low[u], dfs_low[v.first]);       // update dfs_low[u]
    }
    else if (v.first != dfs_parent[u])       // a back edge and not direct cycle
      dfs_low[u] = min(dfs_low[u], dfs_num[v.first]);       // update dfs_low[u]
} }

//inside int main()
  printThis("Articulation Points & Bridges (the input graph must be UNDIRECTED)");
  dfsNumberCounter = 0; dfs_num.assign(V, DFS_WHITE); dfs_low.assign(V, 0);
  dfs_parent.assign(V, -1); articulation_vertex.assign(V, 0);
  printf("Bridges:\n");
  for (int i = 0; i < V; i++)
    if (dfs_num[i] == DFS_WHITE) {
      dfsRoot = i; rootChildren = 0;
      articulationPointAndBridge(i);
      articulation_vertex[dfsRoot] = (rootChildren > 1); }       // special case
  printf("Articulation Points:\n");
  for (int i = 0; i < V; i++)
    if (articulation_vertex[i])
      printf(" Vertex %d\n", i);
\end{lstlisting}
\newpage

\subsection{Tarjan's Algorithm}
\begin{lstlisting}
vi S, visited;                                    // additional global variables
int numSCC;

void tarjanSCC(int u) {
  dfs_low[u] = dfs_num[u] = dfsNumberCounter++;      // dfs_low[u] <= dfs_num[u]
  S.push_back(u);           // stores u in a vector based on order of visitation
  visited[u] = 1;
  for (int j = 0; j < (int)AdjList[u].size(); j++) {
    ii v = AdjList[u][j];
    if (dfs_num[v.first] == DFS_WHITE)
      tarjanSCC(v.first);
    if (visited[v.first])                                // condition for update
      dfs_low[u] = min(dfs_low[u], dfs_low[v.first]);
  }

  if (dfs_low[u] == dfs_num[u]) {         // if this is a root (start) of an SCC
    printf("SCC %d:", ++numSCC);            // this part is done after recursion
    while (1) {
      int v = S.back(); S.pop_back(); visited[v] = 0;
      printf(" %d", v);
      if (u == v) break;
    }
    printf("\n");
} }

//inside int main()
  printThis("Strongly Connected Components (the input graph must be DIRECTED)");
  dfs_num.assign(V, DFS_WHITE); dfs_low.assign(V, 0); visited.assign(V, 0);
  dfsNumberCounter = numSCC = 0;
  for (int i = 0; i < V; i++)
    if (dfs_num[i] == DFS_WHITE)
      tarjanSCC(i);
\end{lstlisting}

\subsection{Bipartite Graph Check}
\begin{lstlisting}
	queue<int> q; q.push(s);
	vi color(V, INF); color[s] = 0;
	bool isBipartite = true;
	while (!q.empty() & isBipartite){
		int u = q.front(); q.pop();
		for (int j = 0; j < (int)AdjList[u].size(); j++){
			ii v = AdjList[u][j];
			if (color[v.first] == INF){
				color[v.first] = 1 - color[u];
				q.push(v.first);}
			else if (color[v.first] == color[u]){
				isBipartite = false; break;}}
	}
\end{lstlisting}

\subsection{Kruskal's Algorithm}
\begin{lstlisting}
  vector< pair<int, ii> > EdgeList;   // (weight, two vertices) of the edge
  for (int i = 0; i < E; i++) {
    scanf("%d %d %d", &u, &v, &w);            // read the triple: (u, v, w)
    EdgeList.push_back(make_pair(w, ii(u, v)));                // (w, u, v)
    AdjList[u].push_back(ii(v, w));
    AdjList[v].push_back(ii(u, w));
  }
  sort(EdgeList.begin(), EdgeList.end()); // sort by edge weight O(E log E)
                      // note: pair object has built-in comparison function

  int mst_cost = 0;
  UnionFind UF(V);                     // all V are disjoint sets initially
  for (int i = 0; i < E; i++) {                      // for each edge, O(E)
    pair<int, ii> front = EdgeList[i];
    if (!UF.isSameSet(front.second.first, front.second.second)) {  // check
      mst_cost += front.first;                // add the weight of e to MST
      UF.unionSet(front.second.first, front.second.second);    // link them
  } }                       // note: the runtime cost of UFDS is very light

  // note: the number of disjoint sets must eventually be 1 for a valid MST
  printf("MST cost = %d (Kruskal's)\n", mst_cost);
\end{lstlisting}

\subsection{Prim's Algorithm}
\begin{lstlisting}
vi taken;                                  // global boolean flag to avoid cycle
priority_queue<ii> pq;            // priority queue to help choose shorter edges

void process(int vtx) {    // so, we use -ve sign to reverse the sort order
  taken[vtx] = 1;
  for (int j = 0; j < (int)AdjList[vtx].size(); j++) {
    ii v = AdjList[vtx][j];
    if (!taken[v.first]) pq.push(ii(-v.second, -v.first));
} }                                // sort by (inc) weight then by (inc) id
// inside int main() --- assume the graph is stored in AdjList, pq is empty
  taken.assign(V, 0);                // no vertex is taken at the beginning
  process(0);   // take vertex 0 and process all edges incident to vertex 0
  mst_cost = 0;
  while (!pq.empty()) {  // repeat until V vertices (E=V-1 edges) are taken
    ii front = pq.top(); pq.pop();
    u = -front.second, w = -front.first;  // negate the id and weight again
    if (!taken[u])                 // we have not connected this vertex yet
      mst_cost += w, process(u); // take u, process all edges incident to u
  }                                        // each edge is in pq only once!
  printf("MST cost = %d (Prim's)\n", mst_cost);
\end{lstlisting}
\newpage

\subsection{Dijkstra's Algorithm}
\begin{lstlisting}
  // Dijkstra routine
  vi dist(V, INF); dist[s] = 0;                    // INF = 1B to avoid overflow
  priority_queue< ii, vector<ii>, greater<ii> > pq; pq.push(ii(0, s));
                             // ^to sort the pairs by increasing distance from s
  while (!pq.empty()) {                                             // main loop
    ii front = pq.top(); pq.pop();     // greedy: pick shortest unvisited vertex
    int d = front.first, u = front.second;
    if (d > dist[u]) continue;   // this check is important, see the explanation
    for (int j = 0; j < (int)AdjList[u].size(); j++) {
      ii v = AdjList[u][j];                       // all outgoing edges from u
      if (dist[u] + v.second < dist[v.first]) {
        dist[v.first] = dist[u] + v.second;                 // relax operation
        pq.push(ii(dist[v.first], v.first));
  } } }  // note: this variant can cause duplicate items in the priority queue
\end{lstlisting}

\subsection{Bellman Ford's Algorithm}
\begin{lstlisting}
  // Bellman Ford routine
  vi dist(V, INF); dist[s] = 0;
  for (int i = 0; i < V - 1; i++)  // relax all E edges V-1 times, overall O(VE)
    for (int u = 0; u < V; u++)                        // these two loops = O(E)
      for (int j = 0; j < (int)AdjList[u].size(); j++) {
        ii v = AdjList[u][j];        // we can record SP spanning here if needed
        dist[v.first] = min(dist[v.first], dist[u] + v.second);         // relax
      }
\end{lstlisting}

\subsection{Check Negative Cycle with Bellman Ford's Algorithm}
\begin{lstlisting}
  bool hasNegativeCycle = false;
  for (int u = 0; u < V; u++)                          // one more pass to check
    for (int j = 0; j < (int)AdjList[u].size(); j++) {
      ii v = AdjList[u][j];
      if (dist[v.first] > dist[u] + v.second)                 // should be false
        hasNegativeCycle = true;     // but if true, then negative cycle exists!
    }
  printf("Negative Cycle Exist? %s\n", hasNegativeCycle ? "Yes" : "No");
\end{lstlisting}

\subsection{Floyd Warshall's Algorithm}
\begin{lstlisting}
  for (int k = 0; k < V; k++) // common error: remember that loop order is k->i->j
    for (int i = 0; i < V; i++)
      for (int j = 0; j < V; j++)
        AdjMatrix[i][j] = min(AdjMatrix[i][j], AdjMatrix[i][k] + AdjMatrix[k][j]);
\end{lstlisting}
\newpage

\subsection{Shortest Path Faster Algorithm}
\begin{lstlisting}
    // SPFA from source S
    // initially, only S has dist = 0 and in the queue
    vi dist(n, INF); dist[S] = 0;
    queue<int> q; q.push(S);
    vi in_queue(n, 0); in_queue[S] = 1;

    while (!q.empty()) {
      int u = q.front(); q.pop(); in_queue[u] = 0;
      for (j = 0; j < (int)AdjList[u].size(); j++) { // all outgoing edges from u
        int v = AdjList[u][j].first, weight_u_v = AdjList[u][j].second;
        if (dist[u] + weight_u_v < dist[v]) { // if can relax
          dist[v] = dist[u] + weight_u_v; // relax
          if (!in_queue[v]) { // add to the queue only if it's not in the queue
            q.push(v);
            in_queue[v] = 1;
          }
        }
      }
    }
\end{lstlisting}

\subsection{Network Flow}
\begin{lstlisting}
void augment(int v, int min_edge){
    if (v == s){
        flow = min_edge;
        return;
    }
    else if (parent[v] != -1){
        int u = parent[v];
        augment(u, min(min_edge, residue[u][v]));
        residue[u][v] -= flow;
        residue[v][u] += flow;
    }
}

void Dinic(){
    max_flow = 0;
    while (true){
        parent.assign(V, -1);
        vector<bool> visited(V, false);
        queue<int> q;
        q.push(s);
        visited[s] = true;
        while (!q.empty()){
            int u = q.front();
            q.pop();
            if (u == t)
                break;
            for (int v : adjList[u])
                if (!visited[v] && residue[u][v] > 0){
                    parent[v] = u;
                    visited[v] = true;
                    q.push(v);
                }
        }

        int new_flow = 0;
        for (int u : adjList[t]){
            if (residue[u][t] <= 0)
                continue;
            flow = 0;
            augment(u, residue[u][t]);
            residue[u][t] -= flow;
            residue[t][u] += flow;
            new_flow += flow;
        }
        if (new_flow == 0)
            break;
        max_flow += new_flow; 
    }
}
\end{lstlisting}

\subsection{Euler Tour}
\begin{lstlisting}
void Euler_tour(int u, list<int> &tour, list<int>::iterator it, 
                vector<vector<pair<int, bool>>> &adj_list) {
    for (auto &edge : adj_list[u]) {
        if (edge.second) {
            int v = edge.first;
            edge.second = false;
            for (auto &bi_edge : adj_list[v]) 
                if (bi_edge.first == u && bi_edge.second) {
                    bi_edge.second = false;
                    break;
                }
            Euler_tour(v, tour, tour.insert(it, u), adj_list);
        }
    }
}
\end{lstlisting}

\subsection{Max Cardinality Bipartite Matching}
\begin{lstlisting}
int augment(int u, const vector<vector<int>> &adj_list, 
            vector<int> &match, vector<bool> &visited) {
    if (visited[u])
        return 0;
    visited[u] = true;
    for (int v : adj_list[u])
        if (match[v] == -1 || augment(match[v], adj_list, match, visited)) {
            match[v] = u;
            return 1;
        }
    return 0;
}   
vector<int> match(adj_list.size(), -1);
vector<bool> visited;
int MCBM = 0;
for (int i = 0; i < adj_list.size(); i++) {
    if (left[i]) {
        visited.assign(adj_list.size(), false);
        MCBM += augment(i, adj_list, match, visited);
    }
}
\end{lstlisting}

\section{Number Theory}
\subsection{Sieve of Eratosthenes}
\begin{lstlisting}
#define BOUND 1000000

bitset<BOUND> bs;
vector<long long> primes;

void sieve() {
    bs.set();
    bs[0] = bs[1] = 0;
    for (long long i = 2; i <= BOUND; i++) {
        if (bs[i]) {
            for (long long j = i * i; j <= BOUND; j += i)
                bs[j] = 0;
            primes.push_back(i);
        }
    }
}

bool is_prime(long long N) {
    if (N <= BOUND)
        return bs[N];
    for (long long prime: primes) {
        if (prime > sqrt(N))
            return true;
        if (N % prime == 0)
            return false;
    }
    return true;
}
\end{lstlisting}
\newpage

\subsection{Prime Factors}
\begin{lstlisting}
vi primeFactors(ll N) {   // remember: vi is vector of integers, ll is long long
  vi factors;                    // vi `primes' (generated by sieve) is optional
  ll PF_idx = 0, PF = primes[PF_idx];     // using PF = 2, 3, 4, ..., is also ok
  while (N != 1 && (PF * PF <= N)) {   // stop at sqrt(N), but N can get smaller
    while (N % PF == 0) { N /= PF; factors.push_back(PF); }    // remove this PF
    PF = primes[++PF_idx];                              // only consider primes!
  }
  if (N != 1) factors.push_back(N);     // special case if N is actually a prime
  return factors;         // if pf exceeds 32-bit integer, you have to change vi
}
\end{lstlisting}

\subsection{Matrix}
\begin{lstlisting}
#define MAX_N 2                                  // increase this if needed
struct Matrix { ll mat[MAX_N][MAX_N]; };     // to let us return a 2D array

Matrix matMul(Matrix a, Matrix b) {            // O(n^3), but O(1) as n = 2
  Matrix ans; int i, j, k;
  for (i = 0; i < MAX_N; i++)
    for (j = 0; j < MAX_N; j++)
      for (ans.mat[i][j] = k = 0; k < MAX_N; k++) {
        ans.mat[i][j] += (a.mat[i][k] % MOD) * (b.mat[k][j] % MOD);
        ans.mat[i][j] %= MOD;             // modulo arithmetic is used here
      }
  return ans;
}

Matrix matPow(Matrix base, int p) {  // O(n^3 log p), but O(log p) as n = 2
  Matrix ans; int i, j;
  for (i = 0; i < MAX_N; i++)
    for (j = 0; j < MAX_N; j++)
      ans.mat[i][j] = (i == j);                  // prepare identity matrix
  while (p) {       // iterative version of Divide & Conquer exponentiation
    if (p & 1)                    // check if p is odd (the last bit is on)
      ans = matMul(ans, base);                                // update ans
    base = matMul(base, base);                           // square the base
    p >>= 1;                                               // divide p by 2
  }
  return ans;
}
\end{lstlisting}

\subsection{Catalan Numbers}
\(Cat(n) = \frac{2n!}{n!\times n! \times (n + 1)} \\ 
Cat(n + 1) = \frac{(2n + 2) \times (2n + 1)}{(n + 2) \times (n + 1)}
\times Cat(n) \)

\subsection{Schr\"oder-Hipparchus Number}
\(S(n) = \frac{1}{n}((6n - 9)S(n - 1) - (n - 3)S(n - 2)) \)

\subsection{Extended Euclid}
\begin{lstlisting}
long long x, y, d;

void extended_Euclid(long long a, long long b) {
    if (b == 0) { x = 1; y = 0; d = a; return;}
    extended_Euclid(b, a % b);
    long long x1 = y, y1 = x - (a / b) * y;
    x = x1;
    y = y1;
}

// Gives ax0 + by0 = d.
// x = x0 + (b/d)n, y = y0 - (a/d)n.
extended_Euclid(a, b);

\end{lstlisting}
\newpage

\section{Computational Geometry}
\begin{lstlisting}
const double PI = acos(-1);

struct Point{
  double x, y;
  Point(double x=0, double y=0):x(x), y(y){}
};

typedef Point Vector;

// Vector + Vector = Vector / Point + Vector = Point
Vector operator + (Vector A, Vector B){
  return Vector(A.x + B.x, A.y + B.y);
}

// Point - Point = Vector
Vector operator - (Point A, Point B){
  return Vector(A.x - B.x, A.y - B.y);
}

Vector operator * (Vector A, double p){
  return Vector(A.x * p, A.y * p);
}

Vector operator / (Vector A, double p){
  return Vector(A.x / p, A.y / p);
}

const double eps = 1e-10;
int dcmp(double x){
  if(fabs(x) < eps)
    return 0;
  return x < 0 ? -1 : 1;
}

bool operator < (const Point& a, const Point& b){
  return dcmp(a.x - b.x) < 0 || (dcmp(a.x-b.x)==0 && dcmp(a.y - b.y) < 0);
}

bool operator == (const Point& a, const Point &b){
  return dcmp(a.x-b.x) == 0 && dcmp(a.y-b.y) == 0;
}

double Dot(Vector A, Vector B){
  return A.x*B.x + A.y*B.y;
}

double Length(Vector A){
  return sqrt(Dot(A,A));
}

// polar angle theta is the counterclockwise angle from the x-axis at which a point in the xy-plane lies
// (-pi, pi]
double angle(Vector v) {
  return atan2(v.y, v.x);
}

// counterclockwise angle from A to B [0, pi]
double Angle(Vector A, Vector B){
  return acos(Dot(A,B)/Length(A)/Length(B));
}

double Cross(Vector A, Vector B){
  return A.x*B.y - A.y*B.x;
}

// counterclockwisely rotate A for rad
Vector Rotate(Vector A, double rad){
  return Vector(A.x*cos(rad)-A.y*sin(rad), A.x*sin(rad)+A.y*cos(rad));
}

// unit normal vector for A(left rotate pi/2) A != 0
Vector Normal(Vector A){
  double L = Length(A);
  return Vector(-A.y/L, A.x/L);
}

// P+tv, Q+tw should have only one intersection, iff Cross(v,w) != 0
Point GetLineIntersection(Point P, Vector v, Point Q, Vector w){
  Vector u = P-Q;
  double t = Cross(w,u)/Cross(v,w);
  return P+v*t;
}

// distance from P to line AB
double DistanceToLine(Point P, Point A, Point B){
  Vector v1 = B-A, v2 = P-A;
  return fabs(Cross(v1,v2))/Length(v1); // if no fabs,then directed distance
}

// distance from P to segment AB
double DistanceToSegment(Point P, Point A, Point B){
  if(A == B)
    return Length(P-A);
  Vector v1 = B-A, v2 = P-A, v3 = P-B;
  if(dcmp(Dot(v1,v2))<0)
    return Length(v2);
  if(dcmp(Dot(v1,v3))>0)
    return Length(v3);
  return fabs(Cross(v1,v2))/Length(v1); // if no fabs,then directed distance
}

Point GetLineProjection(Point P, Point A, Point B){
  Vector v = B-A;
  return A+v*(Dot(v,P-A) / Dot(v,v));
}

// determine segment a1a2 and b1b2 normal intersection (only one intersection, not endpoint)
// if allowing intersecting on endpoints: 
// 1) c1 = c2 = 0: on the same line, probably intersecting
// 2) otherwise, one endpoint on the other segment (Use OnSegment() method)
bool segmentProperIntersection(Point a1, Point a2, Point b1, Point b2){
  double c1 = Cross(a2-a1,b1-a1);
  double c2 = Cross(a2-a1,b2-a1);
  double c3 = Cross(b2-b1,a1-b1);
  double c4 = Cross(b2-b1,a2-b1);
  return dcmp(c1)*dcmp(c2)<0 && dcmp(c3)*dcmp(c4)<0;
}

// determine P on segment a1a2 (endpoint excluded)
bool OnSegment(Point p, Point a1, Point a2) {
  return dcmp(Cross(a1-p,a2-p))==0 && dcmp(Dot(a1-p,a2-p))<0;
}

// calulate the direct area for polygon(not necessarily convex)
double PolygonArea(Point* p, int n) {
  double area = 0;
  for(int i=1;i<n-1;i++)
    area += Cross(p[i]-p[0],p[i+1]-p[0]);
  return area/2;
}

// convex hull: n points in array p, ch array for output, return the number of points on hull
// no duplicate points in input; the order of input points is not preserved
// if want input points on edges of hull, change two <= to <
int ConvexHull(Point* p, int n, Point* ch) {
  sort(p,p+n);
  int m = 0;
  for(int i=0;i<n;i++){
    while(m>1 && dcmp(Cross(ch[m-1]-ch[m-2], p[i]-ch[m-2])) <= 0)
      m--;
    ch[m++] = p[i];
  }
  int k = m;
  for(int i=n-2;i>=0;i--){
    while(m>k && dcmp(Cross(ch[m-1]-ch[m-2], p[i]-ch[m-2])) <= 0)
      m--;
    ch[m++] = p[i];
  }
  if(n>1)
    m--;
  return m;
}

// return the diameter of set of points (Rotating Calipers Algorithm) 
// ch: already convex hull (no three points in a line) n: the number of points
double diameter(Point* ch, int n) {
  if(n == 1) return 0;
  if(n == 2) return Length(ch[0] - ch[1]);
  ch[n] = ch[0];
  double ans = 0;
  for(int u = 0, v = 1; u < n; u++) {
    // line for p[u]-p[u+1]
    for(;;) {
      // when Area(p[u], p[u+1], p[v+1]) <= Area(p[u], p[u+1], p[v]) stop rotating
      // aka Cross(p[u+1]-p[u], p[v+1]-p[u]) - Cross(p[u+1]-p[u], p[v]-p[u]) <= 0 (now this angle < pi, no need for abs)
      // from Cross(A,B) - Cross(A,C) = Cross(A,B-C)
      // simplify to Cross(p[u+1]-p[u], p[v+1]-p[v]) <= 0
      double diff = Cross(ch[u+1]-ch[u], ch[v+1]-ch[v]);
      if(dcmp(diff) <= 0) {
        ans = max(ans, Length(ch[u]-ch[v]));
        if(dcmp(diff) == 0)
          ans = max(ans, Length(ch[u]-ch[v+1]));
        break;
      }
      v = (v + 1) % n;
    }
  }
  return ans;
}

// poly: polygon n: the number of points
// return value: (-2, vertex) (-1, edges) (0, outside) (1, inside)
// determine if point on the left side of all edges (vertex already counterclock ordered)
int isPointInPolygon(Point p, Point* poly, int n){
  int wn = 0;
  for(int i=0;i<n;i++){
    if(p == poly[i])
      return -2;
    if(OnSegment(p, poly[i], poly[(i+1)%n]))
      return -1;
    int k = dcmp(Cross(poly[(i+1)%n]-poly[i], p-poly[i]));
    int d1 = dcmp(poly[i].y - p.y);
    int d2 = dcmp(poly[(i+1)%n].y - p.y);
    if(k>0 && d1<=0 && d2>0)
      wn++;
    if(k<0 && d2<=0 && d1>0)
      wn--;
  }
  if(wn != 0)
    return 1;
  return 0;
}

struct Line{
  Point p;
  Vector v;
  Line(Point p, Vector v):p(p),v(v){}
  Point point(double t) {
      return p + v*t;
    }
    Line move(double d) {
      return Line(p + Normal(v)*d, v);
  }
};

struct Circle{
  Point c;
  double r;
  Circle(Point c, double r):c(c),r(r){}
  Point point(double a){
    return Point(c.x + cos(a)*r, c.y + sin(a)*r);
  }
};

// return number of intersection, sol has all intersection
// intersection P = A + t(B-A),simplify to et^2+ft+g = 0
int getLineCircleIntersection(Line L, Circle C, double& t1, double& t2, vector<Point>& sol){
  double a = L.v.x, b = L.p.x - C.c.x, c = L.v.y, d = L.p.y - C.c.y;
  double e = a*a + c*c, f = 2*(a*b+c*d), g = b*b + d*d - C.r*C.r;
  double delta = f*f - 4*e*g;
  if(dcmp(delta) < 0)
    return 0;
  if(dcmp(delta) == 0){
    t1 = t2 = -f / (2*e);
    sol.push_back(L.point(t1));
    return 1;
  }
  t1 = (-f - sqrt(delta)) / (2*e);
  sol.push_back(L.point(t1));
  t2 = (-f + sqrt(delta)) / (2*e);
  sol.push_back(L.point(t2));
  return 2;
}

// return the number of intersection
// if two circle identical, then return -1
int getCircleCircleIntersection(Circle C1, Circle C2, vector<Point>& sol){
  double d = Length(C1.c-C2.c);
  if(dcmp(d) == 0){
    if(dcmp(C1.r-C2.r) == 0)
      return -1;
    return 0;
  }
  if(dcmp(C1.r+C2.r-d) < 0)
    return 0;
  if(dcmp(fabs(C1.r-C2.r) - d) > 0)
    return 0;
  double a = angle(C2.c-C1.c);
  double da = acos((C1.r*C1.r + d*d - C2.r*C2.r) / (2*C1.r*d)); // angle from C1C2 to C1P1
  Point p1 = C1.point(a-da), p2 = C1.point(a+da);
  sol.push_back(p1);
  if(p1 == p2)
    return 1;
  sol.push_back(p2);
  return 2;
}

// tangent lines from P to C
// v[i]: i-th tangent lines, return the number of tangent lines
int getTangents(Point p, Circle C, Vector* v){
  Vector u = C.c - p;
  double dist = Length(u);
  if(dist < C.r)
    return 0;
  else if(dcmp(dist-C.r)==0){
    v[0] = Rotate(u,PI/2);
    return 1;
  } else {
    double ang = asin(C.r / dist);
    v[0] = Rotate(u, -ang);
    v[1] = Rotate(u, +ang);
    return 2;
  }
}

// return the number of tangents, -1 means inf
// a[i], b[i]: point of tangency with i-th tangent on A, B; same when internally or externally tangent
int getTangents(Circle A, Circle B, Point* a, Point* b) {
  int cnt = 0;
  if(A.r < B.r){
    swap(A, B);
    swap(a, b);
  }
  double d2 = (A.c.x-B.c.x)*(A.c.x-B.c.x) + (A.c.y-B.c.y)*(A.c.y-B.c.y);
  double rdiff = A.r - B.r;
  double rsum = A.r + B.r;
  if(dcmp(d2 - rdiff*rdiff) < 0) // containing
    return 0;
  double base = atan2(B.c.y-A.c.y, B.c.x-A.c.x);
  if(dcmp(d2)==0 && dcmp(A.r-B.r)==0) // infinite tangents
    return -1;
  if(dcmp(d2-rdiff*rdiff) == 0){ // inscribe, one tangent
    a[cnt] = A.point(base);
    b[cnt] = B.point(base);
    cnt++;
    return 1;
  }
  double ang = acos((A.r-B.r)/sqrt(d2)); // two external common tangents
  a[cnt] = A.point(base + ang);
  b[cnt] = B.point(base + ang);
  cnt++;
  a[cnt] = A.point(base - ang);
  b[cnt] = B.point(base - ang);
  cnt++;
  if(dcmp(d2-rsum*rsum) == 0){
    a[cnt] = A.point(base);
    b[cnt] = B.point(PI + base);
    cnt++;
  }
  else if(dcmp(d2 - rsum*rsum) > 0){ // two internal common tangents
    double ang = acos((A.r+B.r) / sqrt(d2));
    a[cnt] = A.point(base+ang);
    b[cnt] = B.point(PI+base+ang);
    cnt++;
    a[cnt] = A.point(base-ang);
    b[cnt] = B.point(PI+base-ang);
    cnt++;
  }
  return cnt;
}
\end{lstlisting}

\newpage
\section{String Processing}
\subsection{KMP}
\begin{lstlisting}
#define MAX_N 100010

char T[MAX_N], P[MAX_N]; // T = text, P = pattern
int b[MAX_N], n, m; // b = back table, n = length of T, m = length of P

void kmpPreprocess() { // call this before calling kmpSearch()
  int i = 0, j = -1; b[0] = -1; // starting values
  while (i < m) { // pre-process the pattern string P
    while (j >= 0 && P[i] != P[j]) j = b[j]; // if different, reset j using b
    i++; j++; // if same, advance both pointers
    b[i] = j; // observe i = 8, 9, 10, 11, 12 with j = 0, 1, 2, 3, 4
} }           // in the example of P = "SEVENTY SEVEN" above

void kmpSearch() { // this is similar as kmpPreprocess(), but on string T
  int i = 0, j = 0; // starting values
  while (i < n) { // search through string T
    while (j >= 0 && T[i] != P[j]) j = b[j]; // if different, reset j using b
    i++; j++; // if same, advance both pointers
    if (j == m) { // a match found when j == m
      printf("P is found at index %d in T\n", i - j);
      j = b[j]; // prepare j for the next possible match
} } }
\end{lstlisting}

\subsection{Suffix Array}
\begin{lstlisting}
#define MAX_N 100010                         // second approach: O(n log n)
char T[MAX_N];                   // the input string, up to 100K characters
int n;                                        // the length of input string
int RA[MAX_N], tempRA[MAX_N];        // rank array and temporary rank array
int SA[MAX_N], tempSA[MAX_N];    // suffix array and temporary suffix array
int c[MAX_N];                                    // for counting/radix sort

char P[MAX_N];                  // the pattern string (for string matching)
int m;                                      // the length of pattern string

int Phi[MAX_N];                      // for computing longest common prefix
int PLCP[MAX_N];
int LCP[MAX_N];  // LCP[i] stores the LCP between previous suffix T+SA[i-1]
                                              // and current suffix T+SA[i]

bool cmp(int a, int b) { return strcmp(T + a, T + b) < 0; }      // compare

void constructSA_slow() {               // cannot go beyond 1000 characters
  for (int i = 0; i < n; i++) SA[i] = i; // initial SA: {0, 1, 2, ..., n-1}
  sort(SA, SA + n, cmp); // sort: O(n log n) * compare: O(n) = O(n^2 log n)
}

void countingSort(int k) {                                          // O(n)
  int i, sum, maxi = max(300, n);   // up to 255 ASCII chars or length of n
  memset(c, 0, sizeof c);                          // clear frequency table
  for (i = 0; i < n; i++)       // count the frequency of each integer rank
    c[i + k < n ? RA[i + k] : 0]++;
  for (i = sum = 0; i < maxi; i++) {
    int t = c[i]; c[i] = sum; sum += t;
  }
  for (i = 0; i < n; i++)          // shuffle the suffix array if necessary
    tempSA[c[SA[i]+k < n ? RA[SA[i]+k] : 0]++] = SA[i];
  for (i = 0; i < n; i++)                     // update the suffix array SA
    SA[i] = tempSA[i];
}

void constructSA() {         // this version can go up to 100000 characters
  int i, k, r;
  for (i = 0; i < n; i++) RA[i] = T[i];                 // initial rankings
  for (i = 0; i < n; i++) SA[i] = i;     // initial SA: {0, 1, 2, ..., n-1}
  for (k = 1; k < n; k <<= 1) {       // repeat sorting process log n times
    countingSort(k);  // actually radix sort: sort based on the second item
    countingSort(0);          // then (stable) sort based on the first item
    tempRA[SA[0]] = r = 0;             // re-ranking; start from rank r = 0
    for (i = 1; i < n; i++)                    // compare adjacent suffixes
      tempRA[SA[i]] = // if same pair => same rank r; otherwise, increase r
      (RA[SA[i]] == RA[SA[i-1]] && RA[SA[i]+k] == RA[SA[i-1]+k]) ? r : ++r;
    for (i = 0; i < n; i++)                     // update the rank array RA
      RA[i] = tempRA[i];
    if (RA[SA[n-1]] == n-1) break;               // nice optimization trick
} }

void computeLCP_slow() {
  LCP[0] = 0;                                              // default value
  for (int i = 1; i < n; i++) {                // compute LCP by definition
    int L = 0;                                       // always reset L to 0
    while (T[SA[i] + L] == T[SA[i-1] + L]) L++;      // same L-th char, L++
    LCP[i] = L;
} }

void computeLCP() {
  int i, L;
  Phi[SA[0]] = -1;                                         // default value
  for (i = 1; i < n; i++)                            // compute Phi in O(n)
    Phi[SA[i]] = SA[i-1];    // remember which suffix is behind this suffix
  for (i = L = 0; i < n; i++) {             // compute Permuted LCP in O(n)
    if (Phi[i] == -1) { PLCP[i] = 0; continue; }            // special case
    while (T[i + L] == T[Phi[i] + L]) L++;       // L increased max n times
    PLCP[i] = L;
    L = max(L-1, 0);                             // L decreased max n times
  }
  for (i = 0; i < n; i++)                            // compute LCP in O(n)
    LCP[i] = PLCP[SA[i]];   // put the permuted LCP to the correct position
}

ii stringMatching() {                      // string matching in O(m log n)
  int lo = 0, hi = n-1, mid = lo;              // valid matching = [0..n-1]
  while (lo < hi) {                                     // find lower bound
    mid = (lo + hi) / 2;                              // this is round down
    int res = strncmp(T + SA[mid], P, m);  // try to find P in suffix 'mid'
    if (res >= 0) hi = mid;        // prune upper half (notice the >= sign)
    else          lo = mid + 1;           // prune lower half including mid
  }                                      // observe '=' in "res >= 0" above
  if (strncmp(T + SA[lo], P, m) != 0) return ii(-1, -1);    // if not found
  ii ans; ans.first = lo;
  lo = 0; hi = n - 1; mid = lo;
  while (lo < hi) {            // if lower bound is found, find upper bound
    mid = (lo + hi) / 2;
    int res = strncmp(T + SA[mid], P, m);
    if (res > 0) hi = mid;                              // prune upper half
    else         lo = mid + 1;            // prune lower half including mid
  }                           // (notice the selected branch when res == 0)
  if (strncmp(T + SA[hi], P, m) != 0) hi--;                 // special case
  ans.second = hi;
  return ans;
} // return lower/upperbound as first/second item of the pair, respectively

ii LRS() {                 // returns a pair (the LRS length and its index)
  int i, idx = 0, maxLCP = -1;
  for (i = 1; i < n; i++)                         // O(n), start from i = 1
    if (LCP[i] > maxLCP)
      maxLCP = LCP[i], idx = i;
  return ii(maxLCP, idx);
}

int owner(int idx) { return (idx < n-m-1) ? 1 : 2; }

ii LCS() {                 // returns a pair (the LCS length and its index)
  int i, idx = 0, maxLCP = -1;
  for (i = 1; i < n; i++)                         // O(n), start from i = 1
    if (owner(SA[i]) != owner(SA[i-1]) && LCP[i] > maxLCP)
      maxLCP = LCP[i], idx = i;
  return ii(maxLCP, idx);
}
\end{lstlisting}

\newpage
\section{Config Files}
\subsection{.vimrc}
\begin{lstlisting}
set number
set autoindent
set expandtab
set tabstop=4
set softtabstop=4
set shiftwidth=4
set cursorline
set clipboard=unnamedplus
set cc=100
set ruler
set spell
\end{lstlisting}

\subsection{.bash\_profile}
\begin{lstlisting}
alias vim="gvim -v"
\end{lstlisting}

\end{document}


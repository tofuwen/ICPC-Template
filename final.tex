\documentclass[10pt,landscape]{article}
\usepackage{multicol}
\usepackage{xeCJK}
\usepackage{listings}
\usepackage{color}
\usepackage[left=1.2in,right=1.2in,top=1.2in,bottom=1.2in]{geometry}
\usepackage{titling}
\setlength{\droptitle}{2in}
\definecolor{dkgreen}{rgb}{0,0.6,0}
\definecolor{gray}{rgb}{0.5,0.5,0.5}
\definecolor{mauve}{rgb}{0.58,0,0.82}
\usepackage{rotating}
\usepackage{graphics}
\usepackage{fancyhdr}
\pagestyle{fancy}

\lhead{University of Illinois at Urbana-Champaign}
\rhead{page \thepage}

\lstset{frame=tb,
		language=C++,
		aboveskip=3mm,
		belowskip=3mm,
		showstringspaces=false,
		columns=flexible,
		basicstyle={\small\ttfamily},
		numbers=none,
		%numberstyle=\tiny\color{gray},
		%keywordstyle=\color{dkgreen},
		commentstyle=\color{mauve},
		breaklines=true,
		breakatwhitespace=true,
		tabsize=3
}


% Redefine section commands to use less space
\makeatletter
\renewcommand{\section}{\@startsection{section}{1}{0mm}%
                                {-1ex plus -.5ex minus -.2ex}%
                                {0.5ex plus .2ex}%x
                                {\normalfont\large\bfseries}}
\renewcommand{\subsection}{\@startsection{subsection}{2}{0mm}%
                                {-1explus -.5ex minus -.2ex}%
                                {0.5ex plus .2ex}%
                                {\normalfont\normalsize\bfseries}}
\renewcommand{\subsubsection}{\@startsection{subsubsection}{3}{0mm}%
                                {-1ex plus -.5ex minus -.2ex}%
                                {1ex plus .2ex}%
                                {\normalfont\small\bfseries}}
\makeatother

\begin{document}
\begin{multicols}{2}
\setcounter{page}{1}
\pagenumbering{arabic}
\pagestyle{fancy}
\tableofcontents
\newpage

\section{Data Structures}
\subsection{Bitmasks}
\begin{lstlisting}
#define lowBit(S) (S & (-S))
#define setAll(S, n) (S = (1 << n) - 1)
#define modulo(S, N) ((S) & (N - 1))   // returns S % N, where N is a power of 2
#define isPowerOfTwo(S) (!(S & (S - 1)))
#define nearestPowerOfTwo(S) ((int)pow(2.0, (int)((log((double)S) / log(2.0)) + 0.5)))
#define turnOffLastBit(S) ((S) & (S - 1))
#define turnOnLastZero(S) ((S) | (S + 1))
#define turnOffLastConsecutiveBits(S) ((S) & (S + 1))
#define turnOnLastConsecutiveZeroes(S) ((S) | (S - 1))
\end{lstlisting}

\subsection{Union-Find Disjoint Sets}
\begin{lstlisting}
struct DisjointSets{
	void addelements(int num){
		while (num--)
			s.push_back(-1);
    }
	int find(int elem) {
		return s[elem] < 0 ? elem : s[elem] = find(s[elem]); 
	}
	void setunion(int a, int b) {
		int root1 = find(a), root2 = find(b);
		int newSize = s[root1] + s[root2];
		if (s[root1] <= s[root2]){
			s[root2] = root1;
			s[root1] = newSize;
		}
		else{
			s[root1] = root2;
			s[root2] = newSize;
		}}
    vector<int> s;
};
\end{lstlisting}

\subsection{Segment Tree}
\begin{lstlisting}
// Segment tree for range sum queries.
struct segment_tree {
    vector<long long> st, lazy;
    const vector<long long> &A;
    size_t n;
    inline int left(int p) { return p << 1;}
    inline int right(int p) { return (p << 1) + 1; }
    void propagate(int p, int L, int R) {
        if (lazy[p] != 0) {
            if (L != R) {
                lazy[left(p)] += lazy[p];
                lazy[right(p)] += lazy[p];
            }
            st[p] += (R - L + 1) * lazy[p];
            lazy[p] = 0;
        }}
    void build(int p, int L, int R) {
        if (L == R)
            st[p] = A[L];
        else {
            build(left(p), L, (L + R) / 2);
            build(right(p), (L + R) / 2 + 1, R);
            st[p] = st[left(p)] + st[right(p)];
        }}
    long long update(int p, int L, int R, int i, int j, long long val) {
        propagate(p, L, R);
        if (L > j || R < i)
            return st[p];
        if (L >= i && R <= j) {
            lazy[p] = val;
            propagate(p, L, R);
            return st[p];
        }
        return st[p] = update(left(p), L, (L + R) / 2, i, j, val) + 
                       update(right(p), (L + R) / 2 + 1, R, i, j, val);
    }
    long long query(int p, int L, int R, int i, int j) {
        if (L > j || R < i)
            return 0;
        propagate(p, L, R);
        if (L >= i && R <= j)
            return st[p];
        return query(left(p), L, (L + R) / 2, i, j) +
               query(right(p), (L + R) / 2 + 1, R, i, j);
    }
    segment_tree(const vector<long long> &_A): A(_A) {
        n = A.size();
        st.assign(n * 4, 0);
        lazy.assign(n * 4, 0);
        build(1, 0, n - 1);
    }
    void update(int i, int j, long long val) {
        update(1, 0, n - 1, i, j, val);
    }
    long long query(int i, int j) {
        return query(1, 0, n - 1, i, j);
    }};
\end{lstlisting}

\subsection{Fenwick Tree}
\begin{lstlisting}
#define LSOne(S) (S & (-S))
class FenwickTree {
private:
  vi ft;
public:
  FenwickTree() {}
  // initialization: n + 1 zeroes, ignore index 0
  FenwickTree(int n) { ft.assign(n + 1, 0); }
  int rsq(int b) {                                     // returns RSQ(1, b)
    int sum = 0; for (; b; b -= LSOne(b)) sum += ft[b];
    return sum; }
  int rsq(int a, int b) {                              // returns RSQ(a, b)
    return rsq(b) - (a == 1 ? 0 : rsq(a - 1)); }
  // adjusts value of the k-th element by v (v can be +ve/inc or -ve/dec)
  void adjust(int k, int v) {                    // note: n = ft.size() - 1
    for (; k < (int)ft.size(); k += LSOne(k)) ft[k] += v; }
};
\end{lstlisting}

\subsection{Treap}
\begin{lstlisting}
template<typename T>
struct treap{
    treap(){
        srand(time(0));
        root = nullptr;
    }
    void insert(const T& elem){
        insert(root, elem);    
    }
    void remove(const T& elem){
        remove(root, elem);
    }
    struct node_t{
        T elem;
        shared_ptr<node_t> left, right;
        int priority;
    };

    shared_ptr<node_t> root;
    shared_ptr<node_t> rotateLeft(shared_ptr<node_t> node){
        shared_ptr<node_t> right = node->right, rightLeft = right->left;
        right->left = node;
        node->right = rightLeft;
        return right;
    }

    shared_ptr<node_t> rotateRight(shared_ptr<node_t> node){
        shared_ptr<node_t> left = node->left, leftRight = left->right;
        left->right = node;
        node->left = leftRight;
        return left;
    }
    void insert(shared_ptr<node_t>& node, const T& elem){
        if (node == nullptr){
            node = make_shared<node_t>();
            node->elem = elem;
            node->left = node->right = nullptr;
            node->priority = rand();
            return;
        }
        // We do not allow multiple keys with the same value
        if (node->elem == elem)
            return;

        if (node->elem > elem){
            insert(node->left, elem);
            if (node->priority < node->left->priority)
                node = rotateRight(node);
        }else{
            insert(node->right, elem);
            if (node->priority < node->right->priority)
                node = rotateLeft(node);
        }}
    void remove(shared_ptr<node_t>& node, const T& elem){
        if (node == nullptr)
            return;
        if (node->elem == elem){
            if (!node->left && !node->right)
                node = nullptr;
            // Keep rotating until the node to be deleted becomes a leaf node.
            else if (!node->left || (node->left && node->right && 
                node->left->priority < node->right->priority)){
                node = rotateLeft(node);
                remove(node->left, elem);
            }
            else{
                node = rotateRight(node);
                remove(node->right, elem);
            }
        }
        else if (node->elem > elem)
            remove(node->left, elem);
        else
            remove(node->right, elem);
    }};
\end{lstlisting}

\subsection{Splay}
\begin{lstlisting}
const int maxNodeCnt = 111111;
int nodeCnt, root, type[maxNodeCnt], parent[maxNodeCnt], childs[maxNodeCnt][2],
    size[maxNodeCnt], stack[maxNodeCnt], reversed[maxNodeCnt];
// ...
void clear() {
    root = 0;
    size[0] = 0;
    nodeCnt = 1;
}
int malloc() {
    type[nodeCnt] = 2;
    childs[nodeCnt][0] = childs[nodeCnt][1] = 0;
    size[nodeCnt] = 1;
    reversed[nodeCnt] = 0;
    return nodeCnt ++;
}
void update(int x) {
    size[x] = size[childs[x][0]] + 1 + size[childs[x][1]];
    // ...
}
void pass(int x) {
    // NOTICE: childs[x][i] == 0
    if (reversed[x]) {
        swap(childs[x][0], childs[x][1]);
        type[childs[x][0]] = 0;
        reversed[childs[x][0]] ^= 1;
        type[childs[x][1]] = 1;
        107reversed[childs[x][1]] ^= 1;
        reversed[x] = 0;
    }
    // ...
}
void rotate(int x) {
    int t = type[x],
    y = parent[x],
    z = childs[x][1 - t];
    type[x] = type[y];
    parent[x] = parent[y];
    if (type[x] != 2) {
        childs[parent[x]][type[x]] = x;
    }
    type[y] = 1 - t;
    parent[y] = x;
    childs[x][1 - t] = y;
    if (z) {
        type[z] = t;
        parent[z] = y;
    }
    childs[y][t] = z;
    update(y);
}
void splay(int x) {
    int stackCnt = 0;
    stack[stackCnt ++] = x;
    for (int i = x; type[i] != 2; i = parent[i]) {
        stack[stackCnt ++] = parent[i];
    }
    for (int i = stackCnt - 1; i > -1; -- i) {
        pass(stack[i]);
    }
    while (type[x] != 2) {
        int y = parent[x];
        if (type[x] == type[y]) {
            rotate(y);
        } else {
            rotate(x);
        }
        if (type[x] == 2) {
            break;
            108}
        rotate(x);
    }
    update(x);
}
int find(int x, int rank) {
    while (true) {
        pass(x);
        if (size[childs[x][0]] + 1 == rank) {
            break;
        }
        if (rank <= size[childs[x][0]]) {
            x = childs[x][0];
        } else {
            rank -= size[childs[x][0]] + 1;
            x = childs[x][1];
        }
    }
    return x;
}
void split(int &x, int &y, int a) {
    // NOTICE: x, y != 0
    y = find(x, a + 1);
    splay(y);
    x = childs[y][0];
    type[x] = 2;
    childs[y][0] = 0;
    update(y);
}
void split3(int &x, int &y, int &z, int a, int b) {
    split(x, z, b);
    split(x, y, a - 1);
}
void join(int &x, int y) {
    // NOTICE x, y != 0
    x = find(x, size[x]);
    splay(x);
    childs[x][1] = y;
    type[y] = 1;
    parent[y] = x;
    109update(x);
}
void join3(int &x, int y, int z) {
    join(y, z);
    join(x, y);
}
int getRank(int x) {
    splay(x);
    root = x;
    return size[childs[x][0]];
}
void reverse(int a, int b) {
    int x, y;
    split3(root, x, y, a + 1, b + 1);
    reversed[x] ^= 1;
    join3(root, x, y);
}
\end{lstlisting}

\subsection{Trie}
\begin{lstlisting}
const int maxnode = 4000 * 100 + 10;
const int sigma_size = 26;

// This template use unnecessary large memory.
// should replace ch[maxnode][sigma_size] by vector<node>.
struct Trie {
  int ch[maxnode][sigma_size];
  int val[maxnode];
  int sz; // the number of node
  void clear() { sz = 1; memset(ch[0], 0, sizeof(ch[0])); }
  int idx(char c) { return c - 'a'; }

  // insert string s, with additional information v
  // v has to be non-zero, zero means "this node is not word node" 
  void insert(const char *s, int v) {
    int u = 0, n = strlen(s);
    for(int i = 0; i < n; i++) {
      int c = idx(s[i]);
      if(!ch[u][c]) { // the node not exist
        memset(ch[sz], 0, sizeof(ch[sz]));
        val[sz] = 0;
        ch[u][c] = sz++;
      }
      u = ch[u][c]; // going down
    }
    val[u] = v;
  }};
\end{lstlisting}

\subsection{Kth Element}
\begin{lstlisting}
const int D = 18;
const int N = 100000;
int n, value[N], rank[N], order[D][N], pos[D][N];
long long sum[D][N];
pair <int, int> backup[N];
void build (int d, int l, int r) {
    if (r - l > 1) {
        int m = (l + r) >> 1,
            curLeft = l,
            curRight = m;
        for (int i = l; i < r; ++ i) {
            if (rank[order[d][i]] < m) {
                order[d + 1][curLeft ++] = order[d][i];
            }else{
                order[d + 1][curRight ++] = order[d][i];
            }
            pos[d][i] = curLeft;
        }
        build(d + 1, l, m);
        build(d + 1, m, r);
    }
    sum[d][r - 1] = value[order[d][r - 1]];
    for (int i = r - 2; i >= l; -- i) {
        sum[d][i] = value[order[d][i]] + sum[d][i + 1];
    }
}
// [l, r) [a, b) k-th sum
long long query (int d, int l, int r, int a, int b, int k) {
    if (k) {
        if (r - l == 1) {
            return sum[d][a];
        }
        int m = (l + r) >> 1,
            posBegin = pos[d][a];
        if (rank[order[d][a]] < m) {
            posBegin -= 1;
        }
        int posEnd = pos[d][b - 1],
        posCnt = posEnd - posBegin;
        if (k < posCnt) {
            return query(d + 1, l, m, posBegin, pos[d][b - 1], k);
        }
#define RIGHT(i) m + i + 1 - pos[d][i]
        int rightBegin = RIGHT(a);
        if (rank[order[d][a]] >= m) {
            rightBegin -= 1;
        }
        long long result = (posBegin < m? sum[d + 1][posBegin]: 0) - (posEnd <
                m? sum[d + 1][posEnd]: 0);
        result += query(d + 1, m, r, rightBegin, RIGHT(b - 1), k - posCnt);
#undef RIGHT
        return result;
    }
    return 0;
}
void clear () {
    for (int i = 0; i < n; ++ i) {
        order[0][i] = i;
    }
    build(0, 0, n);
}
int main(){
    std::ios::sync_with_stdio(false);
    int testCount;
    scanf("%d", &testCount);
    for(int t = 1; t <= testCount; ++ t){
        std::cout << "Case #" << t << ":\n";
        scanf("%d", &n);
        for(int i = 0; i < n; ++ i){
            scanf("%d", value + i);
            backup[i] = std::make_pair(value[i], i);
        }
        std::sort(backup, backup + n);
        for(int i = 0; i < n; ++ i){
            rank[backup[i].second] = i;
        }
        int m;
        scanf("%d", &m);
        while(m --){
            int a, b;
            scanf("%d%d", &a, &b);
            b ++;
            int length = b - a;
            long long result = sum[0][a] - (b < n? sum[0][b]: 0);
            result -= query(0, 0, n, a, b, (length + 1) >> 1);
            result -= query(0, 0, n, a, b, length >> 1);
            std::cout << result << "\n";
        }
        std::cout << "\n";
    }
    return 0;
}
\end{lstlisting}

\section{Graph Theory}
\subsection{Articulation Points and Bridges}
\begin{lstlisting}
vi dfs_low;       // additional information for articulation points/bridges/SCCs
vi articulation_vertex;
int dfsNumberCounter, dfsRoot, rootChildren;
int DFS_WHITE = -1; // unvisited

void articulationPointAndBridge(int u) {
  dfs_low[u] = dfs_num[u] = dfsNumberCounter++;      // dfs_low[u] <= dfs_num[u]
  for (int j = 0; j < (int)AdjList[u].size(); j++) {
    ii v = AdjList[u][j];
    if (dfs_num[v.first] == DFS_WHITE) {                          // a tree edge
      dfs_parent[v.first] = u;
      if (u == dfsRoot) rootChildren++;  // special case, count children of root

      articulationPointAndBridge(v.first);

      if (dfs_low[v.first] >= dfs_num[u])              // for articulation point
        articulation_vertex[u] = true;           // store this information first
      if (dfs_low[v.first] > dfs_num[u])                           // for bridge
        printf(" Edge (%d, %d) is a bridge\n", u, v.first);
      dfs_low[u] = min(dfs_low[u], dfs_low[v.first]);       // update dfs_low[u]
    }
    else if (v.first != dfs_parent[u])       // a back edge and not direct cycle
      dfs_low[u] = min(dfs_low[u], dfs_num[v.first]);       // update dfs_low[u]
} }

//inside int main()
  printThis("Articulation Points & Bridges (the input graph must be UNDIRECTED)");
  dfsNumberCounter = 0; dfs_num.assign(V, DFS_WHITE); dfs_low.assign(V, 0);
  dfs_parent.assign(V, -1); articulation_vertex.assign(V, 0);
  printf("Bridges:\n");
  for (int i = 0; i < V; i++)
    if (dfs_num[i] == DFS_WHITE) {
      dfsRoot = i; rootChildren = 0;
      articulationPointAndBridge(i);
      articulation_vertex[dfsRoot] = (rootChildren > 1); }       // special case
  printf("Articulation Points:\n");
  for (int i = 0; i < V; i++)
    if (articulation_vertex[i])
      printf(" Vertex %d\n", i);
\end{lstlisting}

\subsection{Tarjan's Algorithm}
\begin{lstlisting}
vi S, visited;                                    // additional global variables
int numSCC;

void tarjanSCC(int u) {
  dfs_low[u] = dfs_num[u] = dfsNumberCounter++;      // dfs_low[u] <= dfs_num[u]
  S.push_back(u);           // stores u in a vector based on order of visitation
  visited[u] = 1;
  for (int j = 0; j < (int)AdjList[u].size(); j++) {
    ii v = AdjList[u][j];
    if (dfs_num[v.first] == DFS_WHITE)
      tarjanSCC(v.first);
    if (visited[v.first])                                // condition for update
      dfs_low[u] = min(dfs_low[u], dfs_low[v.first]);
  }

  if (dfs_low[u] == dfs_num[u]) {         // if this is a root (start) of an SCC
    printf("SCC %d:", ++numSCC);            // this part is done after recursion
    while (1) {
      int v = S.back(); S.pop_back(); visited[v] = 0;
      printf(" %d", v);
      if (u == v) break;
    }
    printf("\n");
} }

//inside int main()
  printThis("Strongly Connected Components (the input graph must be DIRECTED)");
  dfs_num.assign(V, DFS_WHITE); dfs_low.assign(V, 0); visited.assign(V, 0);
  dfsNumberCounter = numSCC = 0;
  for (int i = 0; i < V; i++)
    if (dfs_num[i] == DFS_WHITE)
      tarjanSCC(i);
\end{lstlisting}

\subsection{Bipartite Graph Check}
\begin{lstlisting}
	queue<int> q; q.push(s);
	vi color(V, INF); color[s] = 0;
	bool isBipartite = true;
	while (!q.empty() & isBipartite){
		int u = q.front(); q.pop();
		for (int j = 0; j < (int)AdjList[u].size(); j++){
			ii v = AdjList[u][j];
			if (color[v.first] == INF){
				color[v.first] = 1 - color[u];
				q.push(v.first);}
			else if (color[v.first] == color[u]){
				isBipartite = false; break;}}
	}
\end{lstlisting}

\subsection{Kruskal's Algorithm}
\begin{lstlisting}
  vector< pair<int, ii> > EdgeList;   // (weight, two vertices) of the edge
  for (int i = 0; i < E; i++) {
    scanf("%d %d %d", &u, &v, &w);            // read the triple: (u, v, w)
    EdgeList.push_back(make_pair(w, ii(u, v)));                // (w, u, v)
    AdjList[u].push_back(ii(v, w));
    AdjList[v].push_back(ii(u, w));
  }
  sort(EdgeList.begin(), EdgeList.end()); // sort by edge weight O(E log E)
                      // note: pair object has built-in comparison function
  int mst_cost = 0;
  UnionFind UF(V);                     // all V are disjoint sets initially
  for (int i = 0; i < E; i++) {                      // for each edge, O(E)
    pair<int, ii> front = EdgeList[i];
    if (!UF.isSameSet(front.second.first, front.second.second)) {  // check
      mst_cost += front.first;                // add the weight of e to MST
      UF.unionSet(front.second.first, front.second.second);    // link them
  } }                       // note: the runtime cost of UFDS is very light

  // note: the number of disjoint sets must eventually be 1 for a valid MST
  printf("MST cost = %d (Kruskal's)\n", mst_cost);
\end{lstlisting}

\subsection{Prim's Algorithm}
\begin{lstlisting}
vi taken;                                  // global boolean flag to avoid cycle
priority_queue<ii> pq;            // priority queue to help choose shorter edges

void process(int vtx) {    // so, we use -ve sign to reverse the sort order
  taken[vtx] = 1;
  for (int j = 0; j < (int)AdjList[vtx].size(); j++) {
    ii v = AdjList[vtx][j];
    if (!taken[v.first]) pq.push(ii(-v.second, -v.first));
} }                                // sort by (inc) weight then by (inc) id
// inside int main() --- assume the graph is stored in AdjList, pq is empty
  taken.assign(V, 0);                // no vertex is taken at the beginning
  process(0);   // take vertex 0 and process all edges incident to vertex 0
  mst_cost = 0;
  while (!pq.empty()) {  // repeat until V vertices (E=V-1 edges) are taken
    ii front = pq.top(); pq.pop();
    u = -front.second, w = -front.first;  // negate the id and weight again
    if (!taken[u])                 // we have not connected this vertex yet
      mst_cost += w, process(u); // take u, process all edges incident to u
  }                                        // each edge is in pq only once!
  printf("MST cost = %d (Prim's)\n", mst_cost);
\end{lstlisting}

\subsection{Dijkstra's Algorithm}
\begin{lstlisting}
  // Dijkstra routine
  vi dist(V, INF); dist[s] = 0;                    // INF = 1B to avoid overflow
  priority_queue< ii, vector<ii>, greater<ii> > pq; pq.push(ii(0, s));
                             // ^to sort the pairs by increasing distance from s
  while (!pq.empty()) {                                             // main loop
    ii front = pq.top(); pq.pop();     // greedy: pick shortest unvisited vertex
    int d = front.first, u = front.second;
    if (d > dist[u]) continue;   // this check is important, see the explanation
    for (int j = 0; j < (int)AdjList[u].size(); j++) {
      ii v = AdjList[u][j];                       // all outgoing edges from u
      if (dist[u] + v.second < dist[v.first]) {
        dist[v.first] = dist[u] + v.second;                 // relax operation
        pq.push(ii(dist[v.first], v.first));
  } } }  // note: this variant can cause duplicate items in the priority queue
\end{lstlisting}

\subsection{Bellman Ford's Algorithm}
\begin{lstlisting}
  // Bellman Ford routine
  vi dist(V, INF); dist[s] = 0;
  for (int i = 0; i < V - 1; i++)  // relax all E edges V-1 times, overall O(VE)
    for (int u = 0; u < V; u++)                        // these two loops = O(E)
      for (int j = 0; j < (int)AdjList[u].size(); j++) {
        ii v = AdjList[u][j];        // we can record SP spanning here if needed
        dist[v.first] = min(dist[v.first], dist[u] + v.second);         // relax
      }
\end{lstlisting}

\subsection{Floyd Warshall's Algorithm}
\begin{lstlisting}
  for (int k = 0; k < V; k++) // common error: remember that loop order is k->i->j
    for (int i = 0; i < V; i++)
      for (int j = 0; j < V; j++)
        AdjMatrix[i][j] = min(AdjMatrix[i][j], AdjMatrix[i][k] + AdjMatrix[k][j]);
\end{lstlisting}

\subsection{Shortest Path Faster Algorithm}
\begin{lstlisting}
    // SPFA from source S
    // initially, only S has dist = 0 and in the queue
    vi dist(n, INF); dist[S] = 0;
    queue<int> q; q.push(S);
    vi in_queue(n, 0); in_queue[S] = 1;

    while (!q.empty()) {
      int u = q.front(); q.pop(); in_queue[u] = 0;
      for (j = 0; j < (int)AdjList[u].size(); j++) { // all outgoing edges from u
        int v = AdjList[u][j].first, weight_u_v = AdjList[u][j].second;
        if (dist[u] + weight_u_v < dist[v]) { // if can relax
          dist[v] = dist[u] + weight_u_v; // relax
          if (!in_queue[v]) { // add to the queue only if it's not in the queue
            q.push(v);
            in_queue[v] = 1;
          }}}}
\end{lstlisting}

\subsection{Network Flow}
\begin{lstlisting}
void augment(int v, int min_edge){
    if (v == s){
        flow = min_edge;
        return;
    }
    else if (parent[v] != -1){
        int u = parent[v];
        augment(u, min(min_edge, residue[u][v]));
        residue[u][v] -= flow;
        residue[v][u] += flow;
    }}
void Dinic(){
    max_flow = 0;
    while (true){
        parent.assign(V, -1);
        vector<bool> visited(V, false);
        queue<int> q;
        q.push(s);
        visited[s] = true;
        while (!q.empty()){
            int u = q.front();
            q.pop();
            if (u == t)
                break;
            for (int v : adjList[u])
                if (!visited[v] && residue[u][v] > 0){
                    parent[v] = u;
                    visited[v] = true;
                    q.push(v);
                }}
        int new_flow = 0;
        for (int u : adjList[t]){
            if (residue[u][t] <= 0)
                continue;
            flow = 0;
            augment(u, residue[u][t]);
            residue[u][t] -= flow;
            residue[t][u] += flow;
            new_flow += flow;
        }
        if (new_flow == 0)
            break;
        max_flow += new_flow; 
    }}
\end{lstlisting}

\subsection{Euler Tour}
\begin{lstlisting}
void Euler_tour(int u, list<int> &tour, list<int>::iterator it, 
                vector<vector<pair<int, bool>>> &adj_list) {
    for (auto &edge : adj_list[u]) {
        if (edge.second) {
            int v = edge.first;
            edge.second = false;
            for (auto &bi_edge : adj_list[v]) 
                if (bi_edge.first == u && bi_edge.second) {
                    bi_edge.second = false;
                    break;
                }
            Euler_tour(v, tour, tour.insert(it, u), adj_list);
        }}}
\end{lstlisting}

\subsection{Max Cardinality Bipartite Matching}
\begin{lstlisting}
int N, M, P, limit;
#define MAXN 50500
#define MAXE 150500
int pair_left[MAXN], pair_right[MAXN], dist_left[MAXN], dist_right[MAXN];
bool visited[MAXN];
int adjlist[MAXN];
int node[MAXE];
int link[MAXE];
bool BFS() {
    queue<int> q;
    memset(dist_right, -1, sizeof dist_right);
    memset(dist_left, -1, sizeof dist_left);
    for (int i = 0; i < N; i++) {
        if (pair_left[i] == -1) {
            dist_left[i] = 0;
            q.push(i);
        }}
    limit = INT_MAX;
    while (!q.empty()) {
        int u = q.front();
        q.pop();
        if (dist_left[u] > limit)
            break;
        for (int i = adjlist[u]; i != -1; i = link[i]) {
            int v = node[i];
            if (dist_right[v] == -1) {
                dist_right[v] = dist_left[u] + 1;
                if (pair_right[v] == -1)
                    limit = dist_right[v];
                else {
                    dist_left[pair_right[v]] = dist_right[v] + 1;
                    q.push(pair_right[v]);
                }}}}
    return limit != INT_MAX;
}
 
bool DFS(int u) {
    for (int i = adjlist[u]; i != -1; i = link[i]) {
        int v = node[i];
        if (!visited[v] && dist_right[v] == dist_left[u] + 1) {
            visited[v] = true;
            if (pair_right[v] != -1 && dist_right[v] == limit)
                continue;
            if (pair_right[v] == -1 || DFS(pair_right[v])) {
                pair_right[v] = u;
                pair_left[u] = v;
                return true;
            }}}
    return false;
}
 
int main() {
    scanf("%d %d %d", &N, &M, &P);

    memset(pair_left, -1, sizeof pair_left);
    memset(pair_right, -1, sizeof pair_right);
    memset(link, -1, sizeof link);
    memset(adjlist, -1, sizeof adjlist);
 
    for (int i = 0; i < P; i++) {
        int u, v;
        scanf("%d %d", &u, &v);
        node[i] = v - 1;
        link[i] = adjlist[u - 1];
        adjlist[u - 1] = i;
    }
    int matching = 0;
    while (BFS()) {
        memset(visited, 0, sizeof visited);
        for (int i = 0; i < N; i++)
            if (pair_left[i] == -1)
                if (DFS(i))
                    matching++;
    }
    printf("%d\n", matching);
    return 0;
}
\end{lstlisting}

\subsection{Min-Cost Flow}
\begin{lstlisting}
const int inf = 1000000000;
int s, t, node, totalCost;
vector<int> head, dist, vtx, next, c, cost;
vector<bool> vis;
void resize(vector<T> &a, int size, T init) //设大小、初始值
void init(int source, int target, int nodeCount) //初始化,记得清空 
void add(int a,int b,int cc,int cst) //双向加边
void spfa() {
	resize(vis, node, false); resize(dist, node, -inf);
	queue<int> q; q.push(t); vis[t]=true; dist[t]=0;
	while (q.size()) {
		int u = q.front();  	q.pop();
		vis[u] = false;
		for (int p = head[u]; p != -1; p = next[p]) {
			if (c[p ^ 1] && dist[u] + cost[p ^ 1] > dist[vtx[p]]) {
				dist[vtx[p]] = dist[u] + cost[p ^ 1];
				if (!vis[vtx[p]]) {
					vis[vtx[p]] = true; 	q.push(vtx[p]);
					if (dist[q.back()] < dist[q.front()]) {
						swap(q.front(), q.back());
}}}}}}
int dfs(int u, int limit) {
	if (u == t) {
		totalCost += limit * dist[s];
		return limit;
	}
	int current = 0;
	vis[u] = true;
	for (int p = head[u]; p != -1; p = next[p]) {
		if (c[p] && !vis[vtx[p]] && dist[vtx[p]] + cost[p] == dist[u]) {
			int delta = dfs(vtx[p], min(limit - current, c[p]));
			c[p] -= delta;	c[p ^ 1] += delta;
			current += delta;
			if (current == limit) {
				break;
			}
		}
	}
	return current;
}
inline bool adjust() {
	int maxi = -inf;
	for (int i = 0; i < node; ++ i) {
		if (vis[i]) {
			for (int p = head[i]; p != -1; p = next[p]) {
				if (c[p] && !vis[vtx[p]]) {
					assert(dist[vtx[p]] + cost[p] != dist[i]);
					maxi = max(maxi, dist[vtx[p]] + cost[p] - dist[i]);
				}
			}
		}
	}
	if (maxi == -inf) {
		return false;
	}
	for (int i = 0; i < node; ++ i) {
		if (vis[i]) {
			dist[i] += maxi;
		}
	}
	return true;
}
int maxCostFlow() {
	spfa();
	totalCost = 0;
	do{
		do{
			resize(vis, node, false);
		}while (dfs(s, inf));
	}while (adjust());
	return totalCost;
}
\end{lstlisting}

\section{Math}
\subsection{Sieve of Eratosthenes}
\begin{lstlisting}
#define BOUND 1000000
bitset<BOUND> bs;
vector<long long> primes;
void sieve() {
    bs.set();
    bs[0] = bs[1] = 0;
    for (long long i = 2; i <= BOUND; i++) {
        if (bs[i]) {
            for (long long j=i*i;j<=BOUND;j+=i) bs[j] = 0;
            primes.push_back(i);}
    }}
\end{lstlisting}

\subsection{Euler Phi function}
\begin{lstlisting}
int euler_phi(int n){
  int m = (int)sqrt(n+0.5);
  int ans = n;
  for(int i=2;i<=m;i++)
    if(n%i==0){
      ans = ans/i*(i-1);
      while(n%i==0) n /= i;
    }
  if(n>1) ans = ans/n*(n-1);
  return ans;}
void euler_phi_table(int n, int *phi){
  for(int i=2;i<=n;i++) phi[i] = 0;
  phi[1] = 1;
  for(int i=2;i<=n;i++)
    if(!phi[i])
      for(int j=i;j<=n;j+=i){
        if(!phi[j]) phi[j] = j;
        phi[j] = phi[j]/i*(i-1);
}}
\end{lstlisting}

\subsection{GCD mod related (CRT)}
\begin{lstlisting}
//ax+by=gcd(a,b),min abs(x)+abs(y) x,y may be negative
void gcd(LL a, LL b, LL & d, LL & x, LL & y) {
  if(!b) { d = a; x = 1; y = 0; }
  else { gcd(b,a%b,d,y,x); y-=x*(a/b);}}
// calculate inv(a) mod n. If not exist, return -1
LL inv(LL a, LL n) {
  LL d, x, y; gcd(a, n, d, x, y);
  return d == 1 ? (x+n)%n : -1; }
// n functions: x=a[i] (mod m[i]) m[i] co-prime
LL CRT(int n, int * a, int * m) {
  LL M = 1, d, y, x = 0;
  for(int i=0;i<n;i++) M *= m[i];
  for(int i=0;i<n;i++) {
    LL w = M / m[i];
    gcd(m[i], w, d, d, y);
    x = (x + y*w*a[i]) % M;
  }
  return (x+M)%M;}
// solve a^x=b mod n. n prime. If no solution, return -1
int log_mod(int a, int b, int n) {
  int m, v, e = 1;
  m = (int)sqrt(n+0.5);
  v = inv(pow_mod(a, m, n), n);
  map<int, int> x; x[1] = 0;
  for(int i=1;i<m;i++) {
    e = mul_mod(e, a, n);
    if(!x.count(e)) x[e] = i;
  }
  for(int i=0;i<m;i++) {
    if(x.count(b)) return i*m + x[b];
    b = mul_mod(b, v, n);
  }
  return -1;}
\end{lstlisting}

\subsection{Enumerate Combination}
\begin{lstlisting}
const int maxn = 1000;
int com[maxn];
bool next_Com(int num, int k){ //0,1...num-1 choose k
  if(k == 0) return false;
  if(com[k-1]!=num-1){ com[k-1]++; return true;}
  int i;
  for(i=k-1;i>=0;i--)
    if(com[i]!=num-k+i) break;
  if(i==-1) return false;
  com[i]++;
  for(int j=i+1;j<k;j++)
    com[j] = com[i]+(j-i);
  return true; }
void makeFirstCom(int k){
  for(int i=0;i<k;i++) com[i] = i;
}
\end{lstlisting}

\subsection{Gauss Elimination}
\begin{lstlisting}
const int maxn = 110;
typedef double Matrix[maxn][maxn];
// require matrix A invertible
// A is augmented matrix, A[i][n] = bi
// After execution, A[i][n] is the value of i-th variable
void gauss_elimination(Matrix A, int n) {
  int i, j, k, r;
  for (i=0; i<n; i++) {
    r = i;
    for (j=i+1; j<n; j++) {
      if (fabs(A[j][i]) > fabs(A[r][i])) r = j;
    }
    if (r != i)
      for (j=0; j<=n; j++)
        swap(A[r][j], A[i][j]);
    for (j=n; j>=i; j--) 
      for (k=i+1; k<n; ++k)
        A[k][j] -= A[k][i] / A[i][i] * A[i][j];
  }
  for (i=n-1; i>=0; i--) {
    for (j=i+1; j<n; j++)
      A[i][n] -= A[j][n] * A[i][j];
    A[i][n] /= A[i][i];
  }}
\end{lstlisting}

\subsection{FFT}
\begin{lstlisting}
const long double PI = acos(0.0) * 2.0;
typedef complex<double> CD;
inline void FFT(vector<CD> &a, bool inverse) {
  int n = a.size();
  for(int i = 0, j = 0; i < n; i++) {
    if(j > i) swap(a[i], a[j]);
    int k = n;
    while(j & (k >>= 1)) j &= ~k;
    j |= k;
  }
  double pi = inverse ? -PI : PI;
  for(int step = 1; step < n; step <<= 1) {
    double alpha = pi / step;
    for(int k = 0; k < step; k++) {
      CD omegak = exp(CD(0, alpha*k)); 
      for(int Ek = k; Ek < n; Ek += step << 1) { 
        int Ok = Ek + step; 
        CD t = omegak * a[Ok];
        a[Ok] = a[Ek] - t; 
        a[Ek] += t;
      }
    }
  }
  if(inverse)
    for(int i = 0; i < n; i++) a[i] /= n;
}
inline vector<double> operator * (const vector<double>& v1, const vector<double>& v2) {
  int s1 = v1.size(), s2 = v2.size(), S = 2;
  while(S < s1 + s2) S <<= 1;
  vector<CD> a(S,0), b(S,0); 
  for(int i = 0; i < s1; i++) a[i] = v1[i];
  FFT(a, false);
  for(int i = 0; i < s2; i++) b[i] = v2[i];
  FFT(b, false);
  for(int i = 0; i < S; i++) a[i] *= b[i];
  FFT(a, true);
  vector<double> res(s1 + s2 - 1);
  for(int i = 0; i < s1 + s2 - 1; i++) res[i] = a[i].real(); 
  return res;
} // 用FFT实现的快速多项式乘法
\end{lstlisting}

\subsection{Simplex}
\begin{lstlisting}
//输入矩阵a描述线性规划的标准形式。a为m+1行n+1列,其中行0~m-1为不等式
//行m为目标函数(最大化),列0~n-1为变量0~n-1的系数,列n为常数项
//第i个约束为a[i][0]*x[0] + a[i][1]*x[1] + ... <= a[i][n]
//目标为max(a[m][0]*x[0] + a[m][1]*x[1] + ... + a[m][n-1]*x[n-1] - a[m][n])
//注意:变量均有非负约束x[i] >= 0
const int maxm = 500; // 约束数目上限
const int maxn = 500; // 变量数目上限
const double INF = 1e100;
const double eps = 1e-10;
struct Simplex {
  int n; // 变量个数
  int m; // 约束个数
  double a[maxm][maxn]; // 输入矩阵
  int B[maxm], N[maxn]; // 算法辅助变量
  void pivot(int r, int c) {
    swap(N[c], B[r]);
    a[r][c] = 1 / a[r][c];
    for(int j = 0; j <= n; j++) if(j != c) a[r][j] *= a[r][c];
    for(int i = 0; i <= m; i++) if(i != r) {
      for(int j = 0; j <= n; j++) if(j != c) a[i][j] -= a[i][c] * a[r][j];
      a[i][c] = -a[i][c] * a[r][c];
    }
  }
  bool feasible() {
    for(;;) {
      int r, c;
      double p = INF;
      for(int i = 0; i < m; i++) if(a[i][n] < p) p = a[r = i][n];
      if(p > -eps) return true;
      p = 0;
      for(int i = 0; i < n; i++) if(a[r][i] < p) p = a[r][c = i];
      if(p > -eps) return false;
      p = a[r][n] / a[r][c];
      for(int i = r+1; i < m; i++) if(a[i][c] > eps) {
        double v = a[i][n] / a[i][c];
        if(v < p) { r = i; p = v; }
      }
      pivot(r, c);
    }
  }
  //解有界返回1,无解返回0,无界返回-1。b[i]为x[i]的值,ret为目标函数的值
  int simplex(int n, int m, double x[maxn], double& ret) {
    this->n = n;
    this->m = m;
    for(int i = 0; i < n; i++) N[i] = i;
    for(int i = 0; i < m; i++) B[i] = n+i;
    if(!feasible()) return 0;
    for(;;) {
      int r, c;
      double p = 0;
      for(int i = 0; i < n; i++) if(a[m][i] > p) p = a[m][c = i];
      if(p < eps) {
        for(int i = 0; i < n; i++) if(N[i] < n) x[N[i]] = 0;
        for(int i = 0; i < m; i++) if(B[i] < n) x[B[i]] = a[i][n];
        ret = -a[m][n];
        return 1;
      }
      p = INF;
      for(int i = 0; i < m; i++) if(a[i][c] > eps) {
        double v = a[i][n] / a[i][c];
        if(v < p) { r = i; p = v; }
      }
      if(p == INF) return -1;
      pivot(r, c);
    }}
};
\end{lstlisting}

\subsection{Pell Function}
\begin{lstlisting}
//求x^2-ny^2=1的最小正整数根,n不是完全平方数 
p[1]=1;p[0]=0; q[1]=0;q[0]=1; a[2]=(int)(floor(sqrt(n)+1e-7));
g[1]=0;h[1]=1;
for (int i=2;i;++i) {
  g[i]=-g[i-1]+a[i]*h[i-1];   h[i]=(n-sqr(g[i]))/h[i-1];
  a[i+1]=(g[i]+a[2])/h[i];    p[i]=a[i]*p[i-1]+p[i-2];
  q[i]=a[i]*q[i-1]+q[i-2];    检查p[i],q[i]是否为解,如果是,则退出
}
\end{lstlisting}

\subsection{二次剩余}
\begin{lstlisting}
/*a*x^2+b*x+c==0 (mod P) 求0..P-1的根 */
int pDiv2,P,a,b,c,Pb,d;
inline int calc(int x,int Time){
    if (!Time) return 1;  int tmp=calc(x,Time/2);
    tmp=(long long)tmp*tmp%P;
    if (Time&1) tmp=(long long)tmp*x%P;    return tmp;
}
inline int rev(int x){ if (!x) return 0;   return calc(x,P-2);}
inline void Compute(){
    while (1) { b=rand()%(P-2)+2;  if (calc(b,pDiv2)+1==P) return; }
}
int main(){
    srand(time(0)^312314);  int T;
    for (scanf("%d",&T);T;--T) {
        scanf("%d%d%d%d",&a,&b,&c,&P);
        if (P==2)  {
            int cnt=0; for (int i=0;i<2;++i) if ((a*i*i+b*i+c)%P==0) ++cnt;
            printf("%d",cnt);
            for (int i=0;i<2;++i) if ((a*i*i+b*i+c)%P==0) printf(" %d",i);
            puts("");
        }else  {
            int delta=(long long)b*rev(a)*rev(2)%P;
            a=(long long)c*rev(a)%P-sqr( (long long)delta )%P;
            a%=P;a+=P;a%=P;  a=P-a;a%=P;  pDiv2=P/2;
            if (calc(a,pDiv2)+1==P) puts("0");
            else {
                int t=0,h=pDiv2;    while (!(h%2)) ++t,h/=2;
                int root=calc(a,h/2);
                if (t>0) {  Compute();  Pb=calc(b,h); }
                for (int i=1;i<=t;++i) {
                    d=(long long)root*root*a%P;
                    for (int j=1;j<=t-i;++j)  d=(long long)d*d%P;
                    if (d+1==P)  root=(long long)root*Pb%P;
                    Pb=(long long)Pb*Pb%P;
                }
                root=(long long)a*root%P;
                int root1=P-root; root-=delta;
                root%=P; if (root<0) root+=P;
                root1-=delta; root1%=P;  if (root1<0) root1+=P;
                if (root>root1) { t=root;root=root1;root1=t;  }
                if (root==root1) printf("1 %d\n",root);
                else printf("2 %d %d\n",root,root1);
  }}}return 0;  }
\end{lstlisting}

\subsection{Schr\"oder-Hipparchus Number}
\(S(n) = \frac{1}{n}((6n - 9)S(n - 1) - (n - 3)S(n - 2)) \)

\subsection{Catalan Numbers}
\(Cat(n) = \frac{2n!}{n!\times n! \times (n + 1)} \\ 
Cat(n + 1) = \frac{(2n + 2) \times (2n + 1)}{(n + 2) \times (n + 1)}
\times Cat(n) \)

\section{Computational Geometry}
\begin{lstlisting}
struct Point{
  double x, y;
  Point(double x=0, double y=0):x(x), y(y){}
};
typedef Point Vector;
// Vector + Vector = Vector / Point + Vector = Point
Vector operator + (Vector A, Vector B){
  return Vector(A.x + B.x, A.y + B.y);}
// Point - Point = Vector
Vector operator - (Point A, Point B){
  return Vector(A.x - B.x, A.y - B.y);}
Vector operator * (Vector A, double p){
  return Vector(A.x * p, A.y * p);}
Vector operator / (Vector A, double p){
  return Vector(A.x / p, A.y / p);}
const double eps = 1e-10;
int dcmp(double x){
  if(fabs(x) < eps) return 0;
  return x < 0 ? -1 : 1; }
bool operator < (const Point& a, const Point& b){
  return dcmp(a.x - b.x) < 0 || (dcmp(a.x-b.x)==0 && dcmp(a.y - b.y) < 0); }
bool operator == (const Point& a, const Point &b){
  return dcmp(a.x-b.x) == 0 && dcmp(a.y-b.y) == 0; }
double Dot(Vector A, Vector B){
  return A.x*B.x + A.y*B.y; }
double Length(Vector A){
  return sqrt(Dot(A,A)); }
// polar angle theta is the counterclockwise angle from the x-axis at which a point in the xy-plane lies
// (-pi, pi]
double angle(Vector v) {
  return atan2(v.y, v.x); }
// counterclockwise angle from A to B [0, pi]
double Angle(Vector A, Vector B){
  return acos(Dot(A,B)/Length(A)/Length(B)); }
double Cross(Vector A, Vector B){
  return A.x*B.y - A.y*B.x; }
// counterclockwisely rotate A for rad
Vector Rotate(Vector A, double rad){
  return Vector(A.x*cos(rad)-A.y*sin(rad), A.x*sin(rad)+A.y*cos(rad)); }
// unit normal vector for A (left rotate pi/2)  A != 0
Vector Normal(Vector A){
  double L = Length(A);
  return Vector(-A.y/L, A.x/L);}
// P+tv,Q+tw should have only one intersection,iff Cross(v,w)!=0
Point GetLineIntersection(Point P, Vector v, Point Q, Vector w){
  Vector u = P-Q;
  double t = Cross(w,u)/Cross(v,w);
  return P+v*t;}
// distance from P to line AB
double DistanceToLine(Point P, Point A, Point B){
  Vector v1 = B-A, v2 = P-A;
  return fabs(Cross(v1,v2))/Length(v1);}
// distance from P to segment AB
double DistanceToSegment(Point P, Point A, Point B){
  if(A == B) return Length(P-A);
  Vector v1 = B-A, v2 = P-A, v3 = P-B;
  if(dcmp(Dot(v1,v2))<0) return Length(v2);
  if(dcmp(Dot(v1,v3))>0) return Length(v3);
  return fabs(Cross(v1,v2))/Length(v1);}
Point GetLineProjection(Point P, Point A, Point B){
  Vector v = B-A;
  return A+v*(Dot(v,P-A) / Dot(v,v)); }
// determine segment a1a2 and b1b2 normal intersection (only one intersection, not endpoint)
// if allowing intersecting on endpoints: 
// 1) c1 = c2 = 0: on the same line, probably intersecting
// 2) otherwise, one endpoint on the other segment (Use OnSegment() method)
bool segmentProperIntersection(Point a1, Point a2, Point b1, Point b2){
  double c1 = Cross(a2-a1,b1-a1); 
  double c2 = Cross(a2-a1,b2-a1);
  double c3 = Cross(b2-b1,a1-b1); 
  double c4 = Cross(b2-b1,a2-b1);
  return dcmp(c1)*dcmp(c2)<0 && dcmp(c3)*dcmp(c4)<0;}
// determine P on segment a1a2 (endpoint excluded)
bool OnSegment(Point p, Point a1, Point a2) {
  return dcmp(Cross(a1-p,a2-p))==0 && dcmp(Dot(a1-p,a2-p))<0;}
// calulate the direct area for polygon(not necessarily convex)
double PolygonArea(Point* p, int n) {
  double area = 0;
  for(int i=1;i<n-1;i++)
    area += Cross(p[i]-p[0],p[i+1]-p[0]);
  return area/2;}
// convex hull: n points in array p, ch array for output, return the number of points on hull
// no duplicate points in input; the order of input points is not preserved
// if want input points on edges of hull, change two <= to <
int ConvexHull(Point* p, int n, Point* ch) {
  sort(p,p+n); int m = 0;
  for(int i=0;i<n;i++){
    while(m>1 && dcmp(Cross(ch[m-1]-ch[m-2], p[i]-ch[m-2])) <= 0)
      m--;
    ch[m++] = p[i];}
  int k = m;
  for(int i=n-2;i>=0;i--){
    while(m>k && dcmp(Cross(ch[m-1]-ch[m-2], p[i]-ch[m-2])) <= 0)
      m--;
    ch[m++] = p[i];}
  if(n>1) m--;
  return m;}
// return the diameter of set of points (Rotating Calipers Algorithm) 
// ch: already convex hull (no three points in a line) n: the number of points
double diameter(Point* ch, int n) {
  if(n == 1) return 0;
  if(n == 2) return Length(ch[0] - ch[1]);
  ch[n] = ch[0];
  double ans = 0;
  for(int u = 0, v = 1; u < n; u++) {
    for(;;) {
      double diff = Cross(ch[u+1]-ch[u], ch[v+1]-ch[v]);
      if(dcmp(diff) <= 0) {
        ans = max(ans, Length(ch[u]-ch[v]));
        if(dcmp(diff) == 0)
          ans = max(ans, Length(ch[u]-ch[v+1]));
        break;
      } v = (v + 1) % n;
    }}
  return ans;}
// poly: polygon n: the number of points
// return value: (-2, vertex) (-1, edges) (0, outside) (1, inside)
// determine if point on the left side of all edges (vertex already counterclock ordered)
int isPointInPolygon(Point p, Point* poly, int n){
  int wn = 0;
  for(int i=0;i<n;i++){
    if(p == poly[i]) return -2;
    if(OnSegment(p, poly[i], poly[(i+1)%n])) return -1;
    int k = dcmp(Cross(poly[(i+1)%n]-poly[i], p-poly[i]));
    int d1 = dcmp(poly[i].y - p.y);
    int d2 = dcmp(poly[(i+1)%n].y - p.y);
    if(k>0 && d1<=0 && d2>0) wn++;
    if(k<0 && d2<=0 && d1>0) wn--;
  }
  if(wn != 0) return 1;
  return 0;
}
struct Line{
  Point p; Vector v;
  Line(Point p, Vector v):p(p),v(v){}
  Point point(double t) {return p + v*t;}
  Line move(double d) {return Line(p + Normal(v)*d, v);}
};
struct Circle{
  Point c;
  double r;
  Circle(Point c, double r):c(c),r(r){}
  Point point(double a){return Point(c.x + cos(a)*r, c.y + sin(a)*r);}
};
// return number of intersection, sol has all intersection
// intersection P = A + t(B-A),simplify to et^2+ft+g = 0
int getLineCircleIntersection(Line L, Circle C, double& t1, double& t2, vector<Point>& sol){
  double a = L.v.x, b = L.p.x - C.c.x, c = L.v.y, d = L.p.y - C.c.y;
  double e = a*a + c*c, f = 2*(a*b+c*d), g = b*b + d*d - C.r*C.r;
  double delta = f*f - 4*e*g;
  if(dcmp(delta) < 0) return 0;
  if(dcmp(delta) == 0){
    t1 = t2 = -f / (2*e);
    sol.push_back(L.point(t1));
    return 1; }
  t1 = (-f - sqrt(delta)) / (2*e);
  sol.push_back(L.point(t1));
  t2 = (-f + sqrt(delta)) / (2*e);
  sol.push_back(L.point(t2));
  return 2;}
// return the number of intersection
// if two circle identical, then return -1
int getCircleCircleIntersection(Circle C1, Circle C2, vector<Point>& sol){
  double d = Length(C1.c-C2.c);
  if(dcmp(d) == 0){
    if(dcmp(C1.r-C2.r) == 0) return -1;
    return 0;
  }
  if(dcmp(C1.r+C2.r-d) < 0) return 0;
  if(dcmp(fabs(C1.r-C2.r) - d) > 0) return 0;
  double a = angle(C2.c-C1.c);
  double da = acos((C1.r*C1.r + d*d - C2.r*C2.r) / (2*C1.r*d)); // angle from C1C2 to C1P1
  Point p1 = C1.point(a-da), p2 = C1.point(a+da);
  sol.push_back(p1);
  if(p1 == p2) return 1;
  sol.push_back(p2);
  return 2;}
// tangent lines from P to C
// v[i]: i-th tangent lines, return the number of tangent lines
int getTangents(Point p, Circle C, Vector* v){
  Vector u = C.c - p;
  double dist = Length(u);
  if(dist < C.r) return 0;
  else if(dcmp(dist-C.r)==0){
    v[0] = Rotate(u,PI/2);
    return 1;
  } else {
    double ang = asin(C.r / dist);
    v[0] = Rotate(u, -ang); v[1] = Rotate(u, +ang);
    return 2;
  }}
// return the number of tangents, -1 means inf
// a[i], b[i]: point of tangency with i-th tangent on A, B; same when internally or externally tangent
int getTangents(Circle A, Circle B, Point* a, Point* b) {
  int cnt = 0;
  if(A.r < B.r){ swap(A, B); swap(a, b); }
  double d2 = (A.c.x-B.c.x)*(A.c.x-B.c.x) + (A.c.y-B.c.y)*(A.c.y-B.c.y);
  double rdiff = A.r - B.r;
  double rsum = A.r + B.r;
  if(dcmp(d2 - rdiff*rdiff) < 0) // containing
    return 0;
  double base = atan2(B.c.y-A.c.y, B.c.x-A.c.x);
  if(dcmp(d2)==0 && dcmp(A.r-B.r)==0) // infinite tangents
    return -1;
  if(dcmp(d2-rdiff*rdiff) == 0){ // inscribe, one tangent
    a[cnt] = A.point(base); b[cnt] = B.point(base);
    cnt++; return 1;
  }
  double ang = acos((A.r-B.r)/sqrt(d2)); // two external common tangents
  a[cnt] = A.point(base + ang);
  b[cnt] = B.point(base + ang); cnt++;
  a[cnt] = A.point(base - ang);
  b[cnt] = B.point(base - ang); cnt++;
  if(dcmp(d2-rsum*rsum) == 0){
    a[cnt] = A.point(base);
    b[cnt] = B.point(PI + base); cnt++;
  }
  else if(dcmp(d2 - rsum*rsum) > 0){ // two internal common tangents
    double ang = acos((A.r+B.r) / sqrt(d2));
    a[cnt] = A.point(base+ang);
    b[cnt] = B.point(PI+base+ang); cnt++;
    a[cnt] = A.point(base-ang);
    b[cnt] = B.point(PI+base-ang); cnt++;
  }
  return cnt;}

void CircleCenter(point  p0 , point p1 , point p2 , point &cp ){ 
     double a1=p1.x-p0.x , b1=p1.y-p0.y , c1=(sqr(a1)+sqr(b1)) / 2 ;
     double a2=p2.x-p0.x , b2=p2.y-p0.y , c2=(sqr(a2)+sqr(b2)) / 2 ;
     double d = a1*b2 - a2*b1 ;
     cp.x = p0.x + ( c1*b2 - c2*b1 ) / d ;
     cp.y = p0.y + ( a1*c2 - a2*c1 ) / d ;}
double Incenter(point A, point B, point C, point &cp ){ 
  double s , p , r , a , b , c ;
  a = dis(B, C) , b = dis(C, A) , c = dis(A, B) ; p = (a +b +c) / 2 ;
  s = sqrt ( p * ( p-a ) * ( p-b ) * ( p-c ) ) ; r = s / p ;
  cp.x = ( a*A.x + b*B. x + c*C.x ) / ( a + b + c ) ;
  cp.y = ( a*A.y + b*B. y + c*C.y ) / ( a + b + c ) ;
  return r ;}
void Orthocenter(point A, point B, point C, point &cp ){ 
  CircleCenter(A, B, C, cp );
  cp.x = A.x + B.x + C.x - 2 * cp.x ;cp.y = A.y + B.y + C.y - 2 * cp.y ;}

double twoCircleAreaUnion(point a, point b , double r1, double r2){
  if (r1+r2<=(a-b).dist()) return 0;
  if (r1+(a-b).dist()<=r2) return pi*r1*r1;
  if (r2+(a-b).dist()<=r1) return pi*r2*r2;
  double c1, c2, ans=0;
  c1=(r1*r1-r2*r2+(a-b).dis())/(a-b).dist()/r1/2.0;
  c2=(r2*r2-r1*r1+(a-b).dis())/(a-b).dist()/r2/2.0;
  double s1,s2; s1=acos(c1);  s2=acos(c2);
  ans+=s1*r1*r1-r1*r1*sin(s1)*cos(s1);
  ans+=s2*r2*r2-r2*r2*sin(s2)*cos(s2);
  return ans;
}//====两园面积交 dist=是距离,dis是平方

double area2(point pa, point pb) {
  if (pa.len() < pb.len()) swap(pa, pb);  if (pb.len() < eps) return 0;
  double a, b, c, B, C, sinB, cosB, sinC, cosC, S, h, theta;
  a = pb.len();   b = pa.len();   c = (pb-pa).len();
  cosB=dot(pb,pb-pa)/a/c;    sinB=fabs(det(pb,pb-pa)/a/c);  
  cosC=dot(pa, pb) / a / b;   sinC=fabs(det(pa,pb)/a/b);
  B=atan2(sinB , cosB);   C=atan2(sinC, cosC);
  if (a > r) {  S = C/2*r*r;    h = a*b*sinC/c;
    if (h < r && B < PI/2) S -= (acos(h/r)*r*r - h*sqrt(r*r-h*h));
  }
  else if (b > r) { theta = PI - B - asin(sinB/r*a);
    S = .5*a*r*sin(theta) + (C-theta)/2*r*r;  }
  else S = .5*sinC*a*b;  return S; }// a, b, c, r fixed
double area(const point &o) {
  double S = 0; point oa = a-o, ob = b-o, oc = c-o;
  S += area2(oa, ob) * sign(det(oa, ob));
  S += area2(ob, oc) * sign(det(ob, oc));
  S += area2(oc, oa) * sign(det(oc, oa)); return abs(S);
} //=====多边形和圆相交的面积用有向面积,划分成一个三角形和圆的面积的交 

\end{lstlisting}

\begin{lstlisting}
随机增量最小覆盖圆
const double eps=1e-7;
const int maxn=100000;
class circle{
    point o;
    double r;
}
point a[maxn];
int n;
circle ans;
double area(point a, point b, point c){
    return ((b.x-a.x)*(c.y-a.y)-(b.y-a.y)*(c.x-a.x));
}
double dis(point a, point b){
    return (a.x-b.x)*(a.x-b.x) + (a.y-b.y)*(a.y-b.y);
}
void init(){
    int i,j,k;
    scanf("%d",&n);
    rep(i,n) scanf("%lf%lf",&a[i].x,&a[i].y);
}
bool check(const point &a){
    return sqr(a.x-ans.o.x) + sqr(a.y-ans.o.y) <= ans.r + zero;
}
void Mincircle(){
    int i,j,k;
    ans.r=0; ans.x=0; ans.y=0;
    rep(i,n) if (!check(a[i])) {
        ans.o=a[i]; ans.r=0;
        rep(j,i) if (!check(a[j])) {
            CircleCenter(a[i],a[j],ans.o);
            ans.r=dis(ans.o,a[i]);
            rep(k,j) if (!check(a[k])) {
                CircleCenter(a[i],a[j],a[k],ans.o);
                ans.r=dis(ans.o,a[i]);
            }
        }
    }
    printf("%.4lf\n",sqrt(ans.r));
}
\end{lstlisting}

\begin{lstlisting}
半平面交 n^2
const int maxn=200;
const double eps=1e-8;
const int infinite=10000;
struct point{
    double x,y;
    void input(){
        scanf("%lf%lf",&x,&y);
    }
} sol[maxn],tmp[maxn];
struct Tline{
    point a,b;
} line[maxn];
int n,m;
void rebuild(point a, point b){
    int i,t;
    double k1,k2;
    sol[m]=sol[0]; t=0;
    foru(i,1,m){
        k1=area(a,b,sol[i]);
        k2=area(a,b,sol[i-1]);
        if (cmp(k1)*cmp(k2)<0){
            tmp[t].x=(sol[i].x*k2-sol[i-1].x*k1) / (k2-k1);
            tmp[t].y=(sol[i].y*k2-sol[i-1].y*k1) / (k2-k1);
            t++;
        }
        if (cmp(area(a,b,sol[i])) >=0){
            tmp[t]=sol[i];
            t++;
        }
    }
    m=t;
    rep(i,m) sol[i]=tmp[i];
}
void work(){
    int i,j,k;
    double ans;
    point o;
    sol[0].x = 0; sol[0].y = 0;
    sol[1].x = infinite; sol[1].y = 0;
    sol[2].x = infinite; sol[2].y = infinite;
    sol[3].x = 0; sol[3].y = infinite;
    m=4;
    rep(i,n) rebuild(line[i].a,line[i].b);
    // 保留直线line[i].a,line[i+1].b
    左边的点
        if (m>0) printf("1\n");
        else printf("0\n");
}
\end{lstlisting}

\section{String Processing}
\subsection{KMP}
\begin{lstlisting}
#define MAX_N 100010

char T[MAX_N], P[MAX_N]; // T = text, P = pattern
int b[MAX_N], n, m; // b = back table, n = length of T, m = length of P

void kmpPreprocess() { // call this before calling kmpSearch()
  int i = 0, j = -1; b[0] = -1; // starting values
  while (i < m) { // pre-process the pattern string P
    while (j >= 0 && P[i] != P[j]) j = b[j]; // if different, reset j using b
    i++; j++; // if same, advance both pointers
    b[i] = j; // observe i = 8, 9, 10, 11, 12 with j = 0, 1, 2, 3, 4
} }           // in the example of P = "SEVENTY SEVEN" above

void kmpSearch() { // this is similar as kmpPreprocess(), but on string T
  int i = 0, j = 0; // starting values
  while (i < n) { // search through string T
    while (j >= 0 && T[i] != P[j]) j = b[j]; // if different, reset j using b
    i++; j++; // if same, advance both pointers
    if (j == m) { // a match found when j == m
      printf("P is found at index %d in T\n", i - j);
      j = b[j]; // prepare j for the next possible match
} } }
\end{lstlisting}

\subsection{Suffix Array}
\begin{lstlisting}
#define MAX_N 100010          // second approach: O(n log n)
char T[MAX_N];         // the input string, up to 100K characters
int n;                              // the length of input string
int RA[MAX_N], tempRA[MAX_N]; // rank array and temporary rank array
int SA[MAX_N], tempSA[MAX_N];  // suffix array and temporary suffix array
int c[MAX_N];                      // for counting/radix sort

char P[MAX_N];        // the pattern string (for string matching)
int m;               // the length of pattern string

int Phi[MAX_N];       // for computing longest common prefix
int PLCP[MAX_N];
int LCP[MAX_N];  // LCP[i] stores the LCP between previous suffix T+SA[i-1]
                                              // and current suffix T+SA[i]

bool cmp(int a, int b) { return strcmp(T + a, T + b) < 0; }      // compare

void constructSA_slow() {               // cannot go beyond 1000 characters
  for (int i = 0; i < n; i++) SA[i] = i; // initial SA: {0, 1, 2, ..., n-1}
  sort(SA, SA + n, cmp); // sort: O(n log n) * compare: O(n) = O(n^2 log n)
}

void countingSort(int k) {                                          // O(n)
  int i, sum, maxi = max(300, n);   // up to 255 ASCII chars or length of n
  memset(c, 0, sizeof c);                          // clear frequency table
  for (i = 0; i < n; i++)       // count the frequency of each integer rank
    c[i + k < n ? RA[i + k] : 0]++;
  for (i = sum = 0; i < maxi; i++) {
    int t = c[i]; c[i] = sum; sum += t;
  }
  for (i = 0; i < n; i++)          // shuffle the suffix array if necessary
    tempSA[c[SA[i]+k < n ? RA[SA[i]+k] : 0]++] = SA[i];
  for (i = 0; i < n; i++)                     // update the suffix array SA
    SA[i] = tempSA[i];
}

void constructSA() {         // this version can go up to 100000 characters
  int i, k, r;
  for (i = 0; i < n; i++) RA[i] = T[i];                 // initial rankings
  for (i = 0; i < n; i++) SA[i] = i;     // initial SA: {0, 1, 2, ..., n-1}
  for (k = 1; k < n; k <<= 1) {       // repeat sorting process log n times
    countingSort(k);  // actually radix sort: sort based on the second item
    countingSort(0);          // then (stable) sort based on the first item
    tempRA[SA[0]] = r = 0;             // re-ranking; start from rank r = 0
    for (i = 1; i < n; i++)                    // compare adjacent suffixes
      tempRA[SA[i]] = // if same pair => same rank r; otherwise, increase r
      (RA[SA[i]] == RA[SA[i-1]] && RA[SA[i]+k] == RA[SA[i-1]+k]) ? r : ++r;
    for (i = 0; i < n; i++)                     // update the rank array RA
      RA[i] = tempRA[i];
    if (RA[SA[n-1]] == n-1) break;               // nice optimization trick
} }

void computeLCP_slow() {
  LCP[0] = 0;                                       // default value
  for (int i = 1; i < n; i++) {          // compute LCP by definition
    int L = 0;                                 // always reset L to 0
    while (T[SA[i] + L] == T[SA[i-1] + L]) L++;   // same L-th char, L++
    LCP[i] = L;
} }

void computeLCP() {
  int i, L;
  Phi[SA[0]] = -1;        // default value
  for (i = 1; i < n; i++)   // compute Phi in O(n)
    Phi[SA[i]] = SA[i-1];    // remember which suffix is behind this suffix
  for (i = L = 0; i < n; i++) {        // compute Permuted LCP in O(n)
    if (Phi[i] == -1) { PLCP[i] = 0; continue; }  // special case
    while (T[i + L] == T[Phi[i] + L]) L++; // L increased max n times
    PLCP[i] = L;
    L = max(L-1, 0);               // L decreased max n times
  }
  for (i = 0; i < n; i++)         // compute LCP in O(n)
    LCP[i] = PLCP[SA[i]];   // put the permuted LCP to the correct position
}

ii stringMatching() {        // string matching in O(m log n)
  int lo = 0, hi = n-1, mid = lo;    // valid matching = [0..n-1]
  while (lo < hi) {                    // find lower bound
    mid = (lo + hi) / 2;            // this is round down
    int res = strncmp(T + SA[mid], P, m);  // try to find P in suffix 'mid'
    if (res >= 0) hi = mid;        // prune upper half (notice the >= sign)
    else          lo = mid + 1; // prune lower half including mid
  }                          // observe '=' in "res >= 0" above
  if (strncmp(T + SA[lo], P, m) != 0) return ii(-1, -1);    // if not found
  ii ans; ans.first = lo;
  lo = 0; hi = n - 1; mid = lo;
  while (lo < hi) {       // if lower bound is found, find upper bound
    mid = (lo + hi) / 2;
    int res = strncmp(T + SA[mid], P, m);
    if (res > 0) hi = mid;          // prune upper half
    else         lo = mid + 1;       // prune lower half including mid
  }       // (notice the selected branch when res == 0)
  if (strncmp(T + SA[hi], P, m) != 0) hi--;       // special case
  ans.second = hi;
  return ans;
} // return lower/upperbound as first/second item of the pair, respectively

ii LRS() {       // returns a pair (the LRS length and its index)
  int i, idx = 0, maxLCP = -1;
  for (i = 1; i < n; i++)   // O(n), start from i = 1
    if (LCP[i] > maxLCP)
      maxLCP = LCP[i], idx = i;
  return ii(maxLCP, idx);
}

int owner(int idx) { return (idx < n-m-1) ? 1 : 2; }

ii LCS() {       // returns a pair (the LCS length and its index)
  int i, idx = 0, maxLCP = -1;
  for (i = 1; i < n; i++)          // O(n), start from i = 1
    if (owner(SA[i]) != owner(SA[i-1]) && LCP[i] > maxLCP)
      maxLCP = LCP[i], idx = i;
  return ii(maxLCP, idx);
}
\end{lstlisting}

\section{Sundry}
\subsection{Day of some Date}
\begin{lstlisting}
int whatday(int d, int m, int y) { //day month year
    int ans;
    if (m == 1 || m == 2) { m += 12; y --; }
    if ((y < 1752) || (y == 1752 && m < 9)||(y == 1752 && m == 9 && d < 3))
        ans = (d + 2*m + 3*(m+1)/5 + y + y/4 +5) % 7;
    else
        ans = (d + 2*m + 3*(m+1)/5 + y + y/4 - y/100 + y/400)%7;
    return ans;
}
\end{lstlisting}

\subsection{Java}
\begin{lstlisting}
import java.io.*;
import java.util.*;
import java.math.*;
import static java.lang.Math.*;
public class main{
    public static StringTokenizer st;
    public static DataInputStream in;
    public static PrintStream out;
    private static int nextInt() throws Exception{
        for (;st.countTokens()==0;) st=new StringTokenizer(in.readLine());
        return Integer.valueOf(st.nextToken());
    }
    public static BigInteger getsqrt(BigInteger n){
        if (n.compareTo(BigInteger.ZERO)<=0) return n;
        BigInteger x,xx,txx;
        xx=x=BigInteger.ZERO;
        for (int t=n.bitLength()/2;t>=0;t--){
            txx=xx.add(x.shiftLeft(t+1)).add(BigInteger.ONE.shiftLeft(t+t));
            if (txx.compareTo(n)<=0){
                x=x.add(BigInteger.ONE.shiftLeft(t));
                xx=txx;
            }
        }
        return x;
    }
    public static void main(String args[]) throws Exception{
        in=new DataInputStream(System.in);
        out=new PrintStream(new BufferedOutputStream(System.out));
        st=new StringTokenizer(in.readLine());
        out.close();
    }
}

\end{lstlisting}
\end{multicols}
\end{document}

